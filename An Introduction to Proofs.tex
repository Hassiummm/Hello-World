% chktex-file 1 chktex-file 8 chktex-file 9 chktex-file 12 chktex-file 13 chktex-file 15 chktex-file 17 chktex-file 18 chktex-file 26 chktex-file 31 chktex-file 36 chktex-file 44 
\documentclass[10pt]{article}
%%% This is a scheme of a simple package. %%%
% version 1.0.6

% 1. Packages
\usepackage[T1]{fontenc}
\usepackage[hidelinks]{hyperref}
\usepackage[explicit]{titlesec}
\usepackage[utf8]{inputenc}
\usepackage{amsmath,amsthm,amssymb,amsfonts,mathrsfs,mathtools,nicematrix,chngcntr,centernot,ytableau,tikz-cd}
\usepackage{environ,setspace,geometry,enumerate,enumitem,blindtext,multicol,xcolor,calligra,graphicx,wrapfig,pgfplots,mdframed,tabularx,lipsum,comment,csquotes}
\usepackage{chemfig}

% 2. General Commands

% Enable useless warnings
% chktex-file 1 
% chktex-file 36 
% chktex-file 12 
% chktex-file 26 
% chktex-file 18

% Multicolumn setup
\setlength{\columnseprule}{1pt}
\def\columnseprulecolor{\color{black}}

% Geometry 
\geometry{letterpaper, margin=0.75in}
\setstretch{1.25}

% Newsection (shown in ToC, no counter)
\makeatletter
\newcommand\newsection[1]{%
  \section*{#1}%
  \addcontentsline{toc}{section}{#1}%
}
\makeatother

% Backslash
\newcommand{\bs}{\backslash}

% Hyperlink on ToC and section titles
\titleformat{\section}
{\normalfont\Large\bfseries}{\thesection}{1em}{\hyperlink{sec-\thesection}{#1}
\addtocontents{toc}{\protect\hypertarget{sec-\thesection}{}}}
\titleformat{name=\section,numberless}
{\normalfont\Large\bfseries}{}{0pt}{\hyperlink{toc}{#1}}

% Table of contents section only
\setcounter{tocdepth}{1}

% Changefont
\newcommand{\cfd}[1]{\fontfamily{pzc}\selectfont{#1}\fontfamily{cmr}\selectfont{}} 
\newcommand{\cfc}[1]{\calligra{#1}\fontfamily{cmr}\selectfont{}} 

% Enumerate
\setlist[enumerate]{topsep=0pt,itemsep=-1ex,partopsep=1ex,parsep=1ex}

% Remove proofs by empty space
\NewEnviron{reviewmode}{%
    \let\visibleproof\proof
    \let\endvisibleproof\endproof
    \RenewEnviron{proof}{\phantom{}\\\ \\ \\}{}
    \BODY
    \let\proof\visibleproof
    \let\endproof\endvisibleproof
}

% Darkmode (black background, white text)
\newcommand{\darkmode}{\pagecolor{black}\color{white}}

% Enumerate with tab
\newenvironment{level}
{\addtolength{\itemindent}{2em}}
{\addtolength{\itemindent}{-2em}}

% Circle
\newcommand*\circled[1]{\tikz[baseline=(char.base)]{\node[shape=circle,draw,inner sep=0.5pt](char){#1};}}

% Roman numberals
\newcommand{\rom}{\romannumeral}

% Pgfplot setup
\pgfplotsset{compat=1.18}

% 3. Math

% Theorem styles
\theoremstyle{definition}
\newtheorem{definition}{Definition}[section]
\newtheorem{theorem}{Theorem}[section]
\newtheorem*{proposition}{Proposition}
\newtheorem*{lemma}{Lemma}
\newtheorem*{corollary}{Corollary}
\newtheorem*{example}{Example}
\newtheorem*{remark}{Remark}
\newtheorem*{notation}{Notation}
\newtheorem{questioninner}{Exercise}
\newenvironment{exercise}[1][]{%
    \ifx\relax#1\relax\else\renewcommand{\thequestioninner}{#1}\fi 
    \questioninner
}{%
}
\makeatletter % title: exercise, with customize []
\newenvironment{x}[1]{%
    \par\medskip\noindent\textbf{#1\@addpunct{.}}\hskip\labelsep
}{%
    \par 
}
\makeatother % general environment for any title you want

% Equation counter
\counterwithin*{equation}{section}
\counterwithin*{equation}{subsection}

% Quiver (Authors: varkor (https://github.com/varkor), AndréC (https://tex.stackexchange.com/users/138900/andr%C3%A9c))
\usetikzlibrary{calc}
\usetikzlibrary{decorations.pathmorphing}
\tikzset{curve/.style={settings={#1},to path={(\tikztostart)
    .. controls ($(\tikztostart)!\pv{pos}!(\tikztotarget)!\pv{height}!270:(\tikztotarget)$)
    and ($(\tikztostart)!1-\pv{pos}!(\tikztotarget)!\pv{height}!270:(\tikztotarget)$)
    .. (\tikztotarget)\tikztonodes}},
    settings/.code={\tikzset{quiver/.cd,#1}
        \def\pv##1{\pgfkeysvalueof{/tikz/quiver/##1}}},
    quiver/.cd,pos/.initial=0.35,height/.initial=0}
\tikzset{tail reversed/.code={\pgfsetarrowsstart{tikzcd to}}}
\tikzset{2tail/.code={\pgfsetarrowsstart{Implies[reversed]}}}
\tikzset{2tail reversed/.code={\pgfsetarrowsstart{Implies}}}
\tikzset{no body/.style={/tikz/dash pattern=on 0 off 1mm}}

% Change font (math) 
\newcommand{\bb}[1]{\mathbb{#1}}
\newcommand{\ca}[1]{\mathcal{#1}}
\newcommand{\fr}[1]{\mathfrak{#1}}

% Natural, rational, real, complex numbers, integers
\newcommand{\N}{\mathbb{N}}
\newcommand{\C}{\mathbb{C}}
\newcommand{\R}{\mathbb{R}}
\newcommand{\Q}{\mathbb{Q}}
\newcommand{\Z}{\mathbb{Z}}

% Topological space
\newcommand{\T}{\mathcal{T}}

% Fancy symmetric group and alternating group
\newcommand{\Sg}{\mathfrak{S}}
\newcommand{\Ag}{\mathfrak{A}}

% Symmetric group
\DeclareMathOperator{\Sym}{Sym}

% Uniqueness
\newcommand{\uni}{\exists\ \text{!}\ }

% Greek and Hebrew letters
\newcommand{\al}{\alpha}
\newcommand{\be}{\beta}
\newcommand{\ga}{\gamma}
\newcommand{\ep}{\epsilon}
\newcommand{\de}{\delta}
\newcommand{\si}{\sigma}
\newcommand{\la}{\lambda}
\newcommand{\ka}{\kappa}
\newcommand{\vt}{\vartheta}
\newcommand{\vp}{\varphi}
\newcommand{\ve}{\varepsilon}

% Arrows, maps, morphisms
\newcommand{\ua}{\uparrow}
\newcommand{\da}{\downarrow}
\newcommand{\Ra}{\Rightarrow}
\newcommand{\La}{\Leftarrow}
\newcommand{\Ua}{\Uparrow}
\newcommand{\Da}{\Downarrow}
\newcommand{\nRa}{\nRightarrow}
\newcommand{\nLa}{\nLeftarrow}
\newcommand{\hra}{\hookrightarrow}
\newcommand{\hla}{\hookleftarrow}
\newcommand{\lt}{\leadsto}
\newcommand{\mt}{\mapsto}
\newcommand{\rat}{\rightarrowtail}
\newcommand{\lat}{\leftarrowtail}
\DeclareMathOperator{\id}{id}
\newcommand{\bij}{\xrightarrow{\sim}} % right arrow with \sim on it

% Sets, inclusions
\newcommand{\sub}{\subset}
\newcommand{\sube}{\subseteq}
\newcommand{\supe}{\supseteq}
\newcommand{\nsub}{\centernot\subset}
\newcommand{\nsup}{\centernot\supset}
\newcommand{\nsube}{\centernot\subseteq}
\newcommand{\nsupe}{\centernot\supseteq}
\newcommand{\es}{\varnothing}
\newcommand{\sm}{\setminus}
\newcommand{\ps}{\mathscr{P}}

% Otimes, oplus
\newcommand{\ot}{\otimes}
\newcommand{\op}{\oplus}

% Sign of permutation
\DeclareMathOperator{\sgn}{sgn}

% Normal Subgroup
\newcommand{\nsg}{\trianglelefteq}

% Defined as
\newcommand{\defa}{\coloneqq}

% Semidirect product
\newcommand{\sdp}{\rtimes}

% Inverse
\newcommand{\inv}[1]{{#1}^{-1}}

% Union, intersection, disjoint union
\newcommand{\Cap}{\bigcap}
\newcommand{\Cup}{\bigcup}
\newcommand{\DU}{\bigsqcup}

% Modulo
\renewcommand{\mod}{\ \text{mod}\ }

% Conjugacy classes
\DeclareMathOperator{\Cl}{Cl}

% Holomorph
\DeclareMathOperator{\Hol}{Hol}

% Composition of functions
\newcommand{\comp}{~\raisebox{1pt}{\tikz \draw[line width=0.6pt] circle(1.1pt);}~}

% Galois group
\DeclareMathOperator{\gal}{Gal}

% Cardinality
\newcommand{\card}[1]{\lvert{#1}\rvert}

% Norm
\newcommand{\norm}[1]{\lVert{#1}\rVert}

% Partial order
\newcommand{\po}{\preceq}

% Groups generated by
\newcommand{\cyc}[1]{\langle{#1}\rangle}

% Spectrum of a ring
\DeclareMathOperator{\Spec}{Spec}

% Sylow groups
\DeclareMathOperator{\Syl}{Syl}

% Category
\newcommand{\iso}{\approx}
\DeclareMathOperator{\Aut}{Aut}
\DeclareMathOperator{\End}{End}
\DeclareMathOperator{\Hom}{Hom}
\DeclareMathOperator{\Inn}{Inn}
\DeclareMathOperator{\Out}{Out}
\DeclareMathOperator{\Iso}{Iso}
\DeclareMathOperator{\Ob}{Ob}
\newcommand{\cop}[1]{{#1}^{op}}

% Triangle 
\newcommand{\tri}{\triangle}

% Partial derivative
\newcommand{\pa}{\partial}

% 4. Physics & Chemistry

% Quantum: h-bar
\newcommand{\hb}{\hbar}

% Partial operator
\newcommand{\pr}{\partial}



\endinput
\begin{document}
\def\htitle{An Introduction to Proofs}
\def\hauthor{Hassium, Tingyu Wu}
\hsetup
\htoc
\hmain
\section*{Introduction}
wait... tbd

\section{Basic Logic}
Logic is the formal framework and rules of inference that ensure the validity and coherence of arguments in math. 
\begin{remark}
    We shall accept that sentences can be either true or false.
\end{remark} 
\par
A \hdef{proposition} is a sentence that is either true or false in a mathematical system. The label ``true'' or ``false'' assigned to a proposition is called its \hdef{truth value}. We use the letters $T$ and $F$ to represent ``true'' and ``false'', respectively. An \hdef{axiom} is a proposition that is assumed to be true within a mathematical system without requiring proof. Axioms serve as the foundational building blocks of a mathematical theory, from which other propositions can be derived. A \hdef{theorem} is a proposition that has been proven to be true using logical reasoning and the accepted axioms and previously established theorems of the mathematical system. The proof demonstrates why the theorem must hold based on these foundations.
\par
Consider the proposition ``$\pi$ is not a rational number'', which is trivially true. However, we could always find some false companion of this proposition, such as ``$\pi$ is a rational number''. Similarly, we can find a true companion of a false proposition. Let $P$ be a proposition, such companion of $P$ is called the \hdef{negation} of $P$, denoted $\neg P$.
\par
Let $P$ and $Q$ be propositions. Those sentences can be combined using the word ``and'', denoted $P\wedge Q$, and called the \hdef{conjunction} of $P$ and $Q$. The proposition $P\wedge Q$ is true if both $P$ and $Q$ is true. We can combine the propositions by the word ``or'', denoted $P\vee Q$, and called the \hdef{disjunction} of $P$ and $Q$. The proposition $P\vee Q$ is true if at least one of $P$ or $Q$ is true. A \hdef{truth table} is shown below.
\begin{center}
    \begin{tabular}{cc|ccc}
        $P$ & $Q$ & $\neg P$ & $P\wedge Q$ & $P\vee Q$ \\
        \hline
        $T$ & $T$ & $F$ & $T$ & $T$ \\
        $T$ & $F$ & $F$ & $F$ & $T$ \\
        $F$ & $T$ & $T$ & $F$ & $T$ \\
        $F$ & $F$ & $T$ & $F$ & $F$ \\
    \end{tabular}
\end{center}
\par
Two propositions $P$ and $Q$ are \hdef{logically equivalent} if they have the same truth value in every possible combination of truth values for the variables in the statements, denoted $P\equiv Q$.
\begin{example}
    Let $P$ be a proposition, then $P\equiv\neg(\neg P)$ is logically equivalent. To prove this statement, consider $\neg P$ as a proposition $Q$, then we obtain the following truth table.
    \begin{center}
        \begin{tabular}{c|cc}
            $P$ & $Q\equiv\neg P$ & $\neg Q\equiv\neg(\neg P)$ \\
            \hline
            $T$ & $F$ & $T$ \\
            $F$ & $T$ & $F$
        \end{tabular}
    \end{center}
    Here $P$ and $\neg Q$ has the same truth value in each case, so $P\equiv\neg(\neg P)$.
\end{example}
\begin{problem}
    Let $P$, $Q$, and $R$ be propositions. Consider the following statements:
    \begin{enumerate}
        \item $P\vee(Q\vee R)\equiv(P\vee Q)\vee R$;
        \item $P\wedge(Q\wedge R)\equiv(P\wedge Q)\wedge R$.
    \end{enumerate}
    Try to prove or disprove the statements.
\end{problem}
\begin{problem}
    Let $P$, $Q$, and $R$ be propositions. Consider the following statements:
    \begin{enumerate}
        \item $\neg(P\vee Q)\equiv(\neg P)\wedge(\neg Q)$;
        \item $\neg(P\wedge Q)\equiv(\neg P)\vee(\neg Q)$;
        \item $P\vee(Q\wedge R)\equiv(P\vee Q)\wedge(P\vee R)$.
    \end{enumerate}
    Try to prove or disprove the statements. Based on your results, can you find more properties?
\end{problem}
\par
Let $P$ and $Q$ be propositions. Consider the proposition ``if $n$ is a natural number, then $2n$ is an even number''. Let $P$ denotes ``$n$ is a natural number'' and let $Q$ denotes ``$2n$ is an even number'', then the sentence becomes ``if $P$, then $Q$'', denoted $P\implies Q$. This implication called a \hdef{conditional proposition}, $P$ is called the \hdef{antecedent} and $Q$ is called the \hdef{consequent}. The proposition $P\implies Q$ is true if $P$ is true and $Q$ is true. What if $P$ is false? The answer arises from one's intuition.
\begin{example}
    Imagine your high school teacher say ``if you didn't submit your homework, then you haven't completed it''. How would you argue against this sentence? The most likely response would be, ``I did the homework but I didn't submit it''. Whether or not you submitted your homework does not affect the truth value of the implication.
\end{example}
\par
You should convinced by your own intuition (not mine). This case is called a \hdef{vacuous truth}. In the proposition $P\implies Q$, when $P$ is false, $P\implies Q$ is true. The truth table of $P\implies Q$ is shown below.
\begin{center}
    \begin{tabular}{cc|c}
        $P$ & $Q$ & $P\implies Q$ \\
        \hline
        $T$ & $T$ & $T$ \\
        $T$ & $F$ & $F$ \\
        $F$ & $T$ & $T$ \\
        $F$ & $F$ & $T$
    \end{tabular}
\end{center}
\par
Let $P$ and $Q$ be propositions, $(P\implies Q)\wedge(Q\implies P)$ is called a \hdef{biconditional proposition}, denoted $P\iff Q$. We will write this by ``$P$ is true if and only if $Q$ is true''.
\begin{problem}
    Let $P$ and $Q$ be propositions, show $(\neg P\equiv\neg Q)\iff(P\equiv Q)$.
\end{problem}
\begin{example}
    Let $P$ and $Q$ be propositions. Consider the conditional proposition $P\implies Q$. It is false only if $P$ is true and $Q$ is false, that is, $\neg(P\implies Q)\equiv P\wedge(\neg Q)$. Now we take the negation of the right side, $\neg(P\wedge(\neg Q))\equiv(\neg P)\vee(\neg(\neg Q))\equiv(\neg P)\vee Q$.
\end{example}
\begin{problem}
    Write down the truth table of a biconditional proposition. Based on your truth table and the previous example, try to find a proposition $R$ by ``$\vee$'', ``$\wedge$'', and ``$\neg$'' such that $R\equiv(P\iff Q)$. If $P\iff Q$ is true, does $P\equiv Q$?
\end{problem}
\begin{problem}
    Let $P$, $Q$, $R$, and $S$ be propositions. Rewrite $P\implies(Q\implies(R\implies S))$ by `$\vee$'', ``$\wedge$'', and ``$\neg$''. What is the negation of this sentence?
\end{problem}
\begin{problem}
    Let $P$, $Q$, and $R$ be propositions. Try to prove or disprove $P\implies(Q\vee R)\equiv(\neg P)\vee Q\vee R$. What about $P\implies(Q\wedge R)$?
\end{problem}
\par
Given a proposition $P\implies Q$, the \hdef{converse} is defined as $Q\implies P$ and the \hdef{contrapositive} is defined as $(\neg Q)\implies(\neg P)$. The truth table is shown below, and it suffices to conclude that $(P\implies Q)\equiv(\neg Q\implies\neg P)$.
\begin{center}
    \begin{tabular}{cc|ccc}
        $P$ & $Q$ & $P\implies Q$ & $Q\implies P$ & $\neg Q\implies\neg P$ \\
        \hline
        $T$ & $T$ & $T$ & $T$ & $T$ \\
        $T$ & $F$ & $F$ & $T$ & $F$ \\
        $F$ & $T$ & $T$ & $F$ & $T$ \\
        $F$ & $F$ & $T$ & $T$ & $T$
    \end{tabular}
\end{center}
\begin{problem}
    Let $P$ and $Q$ be propositions, when does $(P\implies Q)\equiv(Q\implies P)$?
\end{problem}
\par
Let $P$ be the proposition ``$x$ is a natural number''. Here $x$ is a \hdef{variable}, and the truth value of this proposition depends on $x$. For instance, if $x=1$, then $P$ is true; if $x=0.86$, then $P$ is false. A \hdef{propositional function} is a family of propositions depending on one or more variables. The collection of permitted variables is the \hdef{domain}. Now we write $P(x)$ instead of $P$, so $P(1)$ is true and $P(0.86)$ is false.
\par
Propositional functions are often quantified. The \hdef{universal quantifier} is denoted by ``$\forall$'', and the proposition $\forall x(P(x))$ is true if and only if $P(x)$ is true for every $x$ in its domain. The \hdef{existential quantifier} is denoted by ``$\exists$'', and the proposition $\exists x(P(x))$ is true if and only if $P(x)$ is true for at least one $x$ in its domain. Consider the proposition $\forall x(P(x))$, this means all $x$ make $P(x)$ true, so there does not exists some $x$ such that $P(x)$ is false, which is $\neg(\exists x(\neg P(x)))$.
\begin{example}
    Let $P(x)$ be a proposition, then $\neg(\forall x(P(x)))\iff\neg(\neg(\exists x(\neg P(x))))\iff\exists x(\neg P(x))$. 
\end{example}
\begin{problem}
    Let $P(x)$ be a proposition, show that $\neg(\exists x(P(x)))\iff\forall x(\neg P(x))$.
\end{problem}
\par
The order of quantifiers does matter the meaning of a proposition. Consider the proposition ``for all natural number $x$, there exists a natural number $y$ such that $y>x$''. Pick some $x$, let $y=x+1$, then $y>x$ and $y$ is a natural number, so the proposition is true. However, switching the order of quantifiers gives ``there exists a natural number $y$, for all natural number $x$, $y>x$''. Suppose there exists such $y$, then $y+1$ is a natural number, so let $x=y+1$, it is trivial that $y<x$, hence the proposition is false.
\begin{example}
    Let $P(x)$ and $Q(y)$ be propositions. Consider the proposition $\forall x(\exists y(P(x)\vee Q(y)))$. To find its negation, let $R(x)\equiv\exists y(P(x)\vee Q(y))$, now the negation becomes $\exists x(\neg R(x))$. Since $P$ only depends on $x$, let $S(y)\equiv(P(x)\vee Q(y))$, then we have $\exists x(\neg(\exists y(S(y))))\equiv\exists x(\forall y(\neg S(y)))\equiv\exists x(\forall y(\neg(P(x)\vee Q(y))))\equiv\exists x(\forall y((\neg P(x))\wedge(\neg Q(y))))$.
\end{example}
\begin{problem}
    Let $P(x,y,z)$ be a proposition, consider the following propositions.
    \begin{enumerate}
        \item $Q(x,y,z)\equiv\exists x(\forall y(\forall z(P(x,y,z))))$; %开始写10
        \item $R(x,y,z)\equiv\forall x(\exists y(\forall z(P(x,y,z))))$;
        \item $S(x,y,z)\equiv\forall x(\forall y(\exists z(P(x,y,z))))$.
    \end{enumerate}
    What are the negations of those propositions? What is the negation of $Q\vee(R\wedge S)$?
\end{problem}
\begin{example}
    Let $P(x)$ and $Q(x)$ be propositions. Consider the negation of $P(x)\implies Q(x)$, $\neg(P(x)\implies Q(x))\equiv\neg((\neg P(x))\vee Q(x))\equiv P(x)\wedge(\neg Q(x))\equiv\forall x(P(x)\wedge(\exists x(\neg Q(x))))\equiv\exists x(P(x)\wedge(\neg Q(x)))$. Notice that taking the negation brings an existential quantifier.
\end{example}
\par
In the following sections, we shall assume readers are familiar with basic logic and use it as a tool to understand or prove propositions. Several expressions and their “translations” are shown below.
\begin{center}
    \begin{tabular}{c|c}
        $P\implies Q$ & $P\iff Q$ \\
        \hline
        $P$ implies $Q$; if $P$, then $Q$ & $P$ if and only if $Q$ \\
        $P$ is sufficient for $Q$; $Q$ is necessary for $P$ & $P$ is necessary and sufficient for $Q$
    \end{tabular}
\end{center}
\begin{problem}
    Given the following propositions, analyze their structures.
    \begin{enumerate}
        \item the number $\sqrt{2}$ is not a rational number;
        \item if $x$ is a natural number, then $x$ is an integer;
        \item for all natural number $x$, for all rational number $y$ with $x<y<x+1$, there exists a real number $z$ such that $y<z<y+1$ and $z$ is irrational;
        \item given a sequence $({x}_{n})$ of real numbers, we say $({x}_{n})$ converges to a real number $L$ if, for all real number $\ep>0$, there exists a real number $N$ such that, for all natural number $n$, $n>N$ implies $\card{{x}_{n}-L}<\ep$.
    \end{enumerate} 
    Find the negation of each proposition. 
\end{problem}









\section{Some Axioms of Sets}
\par
In this section, we begin investigating sets, the most basic entities in mathematics. It is natural to ask: What is a set? There is no precise definition of sets. Intuitively, a \hdef{set} is a collection of objects that satisfy some property, and the objects are called \hdef{elements}.
\begin{remark}
    This note is based on the ZFC set theory. In this system, every object is a set and we allow sets of sets. From now on, assume that there exists a set.
\end{remark}
\par
If $S$ is a set and $x$ is an element in $S$, then we say $x$ belongs to $S$, denoted $x\in S$. If $x$ does not belong to $S$, then we write $x\notin S$. If $S$ has no element, then we call it an \hdef{empty set}, denoted $\es$. A set with one element is called a \hdef{singleton}.
\customtheorem{Axiom of extensionality}
\begin{Axiom of extensionality}\index{axiom of extensionality}
    Two sets $A$ and $B$ are equal if and only if they have the same elements.
\end{Axiom of extensionality}
\customtheorem{Axiom schema of separation}
\begin{Axiom schema of separation}\index{axiom schema of separation}
    If $P$ is a property, then for any $X$ there exists a set $Y=\{x\in X\mid P(x)\}$.
\end{Axiom schema of separation}
\par
Elements determine a set. One way to describe a set is to explicitly list the elements. For example, we can write a set $S=\{6,7,8\}$. Another way is to express the elements by the properties they satisfy.
\begin{example}
    Here are several examples of sets.
    \begin{enumerate}
        \item the set $S=\{\es,\{\es\},\{\{\es\}\}\}$ has three elements;
        \item the set $\{2n\mid n\in\N\}$ is the set of all even numbers, where $\N$ is the set of natural numbers;
        \item the set $\Q=\{a/b\mid a,b\in\Z\ \text{and}\ b\ne 0\}$ is the set of rational numbers, where $\Z$ is the set of integers.
    \end{enumerate}
    We shall provide constructions for $\N$ and $\Z$ later.
\end{example}
\begin{problem}
    Write out the set of all positive integers and the set of all prime numbers.
\end{problem}
\begin{definition}
    Let $S$ be a set. A set $R$ is a \hdef{subset} of $S$, denoted $R\sub S$, if for all $x\in R$, $x\in S$. If there exists some $x\in S$ such that $x\notin R$, then $R$ is called a \hdef{proper subset} of $S$, denoted $R\subn S$.
\end{definition}
\par
It suffices to check that axiom schema of separation guarantees that subsets are sets.
\begin{remark}
    Some textbooks use ``$\sube$'' for subset and ``$\sub$'' for proper subsets.
\end{remark}
\begin{proposition}
    Let $A$ be a set, then $A\sub A$.
\end{proposition}
\begin{proof}
    For all $x\in A$, $x\in A$, so $A\sub A$. 
\end{proof}
\begin{proposition}
    Let $X$ and $Y$ be sets, then $X=Y$ if and only $X\sub Y$ and $Y\sub X$.
\end{proposition}
\begin{remark}
    For a biconditional proposition $P\iff Q$, we use the notation ``$(\Ra)$'' in the proof to show $P\implies Q$ and ``$(\La)$'' for $Q\implies P$.
\end{remark}
\begin{proof}
    Let $X$ and $Y$ be sets. $(\Ra)$ For all $x\in X$, since $X=Y$, $x\in Y$, so $X\sub Y$. For all $y\in Y$, since $X=Y$, $y\in X$, so $Y\sub X$. $(\La)$ Suppose $X\ne Y$, then there exist $a\in X$ and $a\notin Y$, so $X\nsub Y$, yet contradiction. 
\end{proof}
\begin{proposition}
    Let $A$ be any set, then $\es\sub A$.
\end{proposition}
\begin{proof}
    Suppose $\es\nsub A$, then there exists $x\in\es$ such that $x\notin A$, since $x\in\es$ is false, contradiction.
\end{proof}
\begin{problem}
    Prove that a set is independent of the order of its elements. For example, $\{1,2,3\}=\{3,2,1\}$.
\end{problem}
\begin{problem}
    If $X$, $Y$, and $Z$ are sets such that $X\sub Y$ and $Y\sub Z$, prove that $X\sub Z$.
\end{problem}
\begin{problem}
    List all the subsets of $X=\{1,2,3\}$, $Y=\{1,2,3,4\}$, and $Z=\{1,\{1,2\},\{2,1\},3\}$.
\end{problem}
Recall that sets are orderless. To construct more complex structures, we need an order between objects. 
\customtheorem{Axiom of pairing}
\begin{Axiom of pairing}\index{axiom of pairing}
    For two objects $a$ and $b$, there exists a set $\{a,b\}$ containing exactly $a$ and $b$.
\end{Axiom of pairing}
\begin{definition}
    Let $a$ and $b$ be some objects. An \hdef{ordered pair} $(a,b)$ is defined as the set $\{\{a\},\{a,b\}\}$.
\end{definition}
\begin{proposition}
    Let $(a,b)$ and $(c,d)$ be ordered pairs, then $(a,b)=(c,d)$ if and only if $a=c$ and $b=d$.
\end{proposition}
\begin{proof}
    We have $(a,b)=\{\{a\},\{a,b\}\}$ and $(c,d)=\{\{c\},\{c,d\}\}$. $(\Ra)$ Suppose $a\ne c$, then $\{a\}\ne\{c\}$. If $\{a\}=\{c,d\}$, then $c=d=a$, yet contradiction. Suppose $b\ne d$. If $a=c$, then $\{a\}=\{c\}$ and $\{a,b\}\ne\{c,d\}$, yet contradiction. $(\La)$ If $a=c$ and $b=d$, then $\{a,b\}=\{c,d\}$ and $\{a\}=\{c\}$, hence $(a,b)=(c,d)$.
\end{proof}
\par
The definition of ordered pairs can be extended to multiple elements. We call $({a}_{1},\dots,{a}_{n})$ a \hdef{n-tuple}.
\begin{problem}
    Prove that $\{a\}=\{a,a\}$.
\end{problem}
\begin{problem}
    Give the definition of a 3-tuple. Determine the number of elements in a 3-tuple.
\end{problem}
\customtheorem{Axiom of union}
\begin{Axiom of union}\index{axiom of union}
    For all set $X$, there exists a set $Y=\Un X$, the union of all elements of $X$.
\end{Axiom of union}
\begin{definition}
    Let $A$ and $B$ be sets. The \hdef{union} of $A$ and $B$ is the set $\{x\mid x\in A\ \text{or}\ x\in B\}$, denoted $A\cup B$. The \hdef{intersection} of $A$ and $B$ is the set $\{x\mid x\in A\ \text{and}\ x\in B\}$. We say $A$ and $B$ are \hdef{disjoint} if $A\cap B=\es$. The \hdef{complement} of $A$ in $B$ is the set $\{x\mid x\in B\ \text{and}\ x\notin A\}$, denoted $B\sm A$.
\end{definition}
\begin{problem}
    Let $A$ and $B$ be sets. Prove that $A\cap B$ and $A\sm B$ are sets based on the axioms.
\end{problem}
\begin{proposition}
    Let $A$ and $B$ be sets, then $A\cup B=B\cup A$.
\end{proposition}
\begin{proof}
    For all $x\in A\cup B$, if $x\in A$, then $x\in B\cup A$; if $x\in B$, then $x\in B$, hence $A\cup B=B\cup A$.
\end{proof}
\begin{problem}
    Let $A$, $B$, and $C$ be sets. Prove the following propositions.
    \begin{enumerate}
        \item $A\cap B=B\cap A$;
        \item $A\cup(B\cup C)=(A\cup B)\cup C$;
        \item $A\cap(B\cap C)=(A\cap B)\cap C$;
        \item $A\cap(B\cup C)=(A\cap B)\cup(A\cap C)$;
        \item $A\cup(B\cap C)=(A\cup B)\cap(A\cup C)$.
    \end{enumerate}
\end{problem}
\begin{theorem}[De Morgan's law]\index{De Morgan's law}
    Let $A$, $B$, and $C$ be sets, then $C\sm(A\cap B)=(C\sm A)\cup(C\sm B)$ and $C\sm(A\cup B)=(C\sm A)\cap(C\sm B)$.
\end{theorem}
\begin{proof}
    Let $x\in C\sm(A\cap B)$, then $x\in C$ and $x\notin A\cap B$, that is, $x\notin A$ and $x\notin B$. If $x\notin C\sm A$, then $x\notin A$, so $x\notin B$ and $x\in C\sm B$. Hence $C\sm(A\cap B)\sub(C\sm A)\cup(C\sm B)$. Now let $x\in(C\sm A)\cup(C\sm B)$, then $x\in C$ and $x\notin A$ or $x\notin B$, so $x\notin A\cap B$, that is, $x\in C\sm(A\cap B)$, hence $(C\sm A)\cup(C\sm B)\sub C\sm(A\cap B)$. The proof of the second part is left as an exercise.
\end{proof}
\par
Some texts assume the existence of an ``universal set'', denoted $U$, which has all objects as elements including itself, so we can define complements of any set $S$ as the set $U\sm S$. However, this assumption leads to a paradox. Consider the set $S$, defined as the set of all sets that are not members of themselves, that is, $S=\{X\mid X\notin X\}$. Does $S$ belong to $S$? This is known as \hdef{Russell's Paradox}. Assume $S\in S$, then by the definition of $S$, $S\in S$ implies $S\notin S$, yet contradiction. Assume $S\notin S$, then $S\in S$. This is also a contradiction. Thus, the existence of such a set $S$ leads to a logical inconsistency.
\begin{definition}
    Let $X$ be a set, and let the \hdef{successor} of $X$ be ${X}^{+}=X\cup\{X\}$. A set $S$ is called an \hdef{inductive set} if $\es\in S$ and for all $X\in S$, ${X}^{+}\in S$.
\end{definition}
\customtheorem{Axiom of infinity}
\begin{Axiom of infinity}\index{axiom of infinity}
    There exists an inductive set.
\end{Axiom of infinity}
\begin{proposition}
    The intersection of two inductive sets is an inductive set.
\end{proposition}
\begin{proof}
    Let $A$ and $B$ be inductive sets, then $\es\in A\cap B$. For all $S\in A\cap B$, $S\in A$ and $S\in B$. Since $A$ and $B$ are inductive, ${S}^{+}\in A$ and ${S}^{+}\in B$, hence $A\cap B$ is inductive.
\end{proof}
\par
\begin{definition}
    The set of all \hdef{natural numbers}, denoted $\N$, is the intersection of all inductive sets. 
\end{definition}
\par
We denote $0=\es$, $1=0+1={0}^{+}$, $2={1}^{+}$, $\dots$
\begin{problem}
    The set of all natural numbers is the smallest inductive set.
\end{problem}
\customtheorem{Axiom of power set}
\begin{Axiom of power set}\index{axiom of power set}
    For any $X$ there exists a set consisting of all subsets of $X$.
\end{Axiom of power set}
\begin{definition}
    Given a set $X$, the set of all subsets of $X$ is called its \hdef{power set}, denoted $\ps(X)$.
\end{definition}
\begin{example}
    Let $X=\{a,b\}$, the power set $\ps(X)=\{\es,\{a\},\{b\},\{a,b\}\}$.
\end{example}
\begin{definition}
    Let $X$ and $Y$ be sets. The \hdef{Cartesian product} $X\times Y$ is the set of all ordered pairs $(a,b)$, where $a\in X$ and $b\in Y$.
\end{definition}
\begin{problem}
    Let $X$ and $Y$ be sets. Write out the set $\ps(\ps(X\cup Y))$. Prove that $X\times Y=\{z\in\ps(\ps(X\cup Y))\mid\text{there exists}\ x\in X\ \text{and}\ y\in Y\ \text{such that}\ z=(x,y)\}\sub\ps(\ps(X\cup Y))$, hence $X\times Y$ is a set.
\end{problem}
\begin{problem}
    Let $S$ be a set, prove that $S\subn\ps(S)$. 
\end{problem}
\begin{problem}
    Let $A$, $B$, and $C$ be sets. Prove the following propositions.
    \begin{enumerate}
        \item $A\times B=B\times A$ if and only if $A=B$;
        \item $A\times(B\times C)=(A\times B)\times C$;
        \item $(A\cup B)\times C=(A\times C)\cup(B\times C)$;
        \item $(A\cap B)\times C=(A\times C)\cap(B\times C)$;
        \item $(A\sm B)\times C=(A\times C)\sm(B\times C)$.
    \end{enumerate}
\end{problem}
\begin{definition}
    A \hdef{binary operation} $R$ is a set of ordered pairs. If $(x,y)\in R$, we write $xRy$. The \hdef{domain} of $R$ is the set $\dom(R)=\{u\mid\text{there exists}\ v\ \text{such that}\ (u,v)\in R\}$. The \hdef{range} of $R$ is the set $\ran(R)=\{v\mid\text{there exists}\ u\ \text{such that}\ (u,v)\in R\}$. 
\end{definition}
\par
It suffices to show a binary operation is indeed a set. Let $R$ be a binary operation, then $R\sub X\times Y$ for some sets $X$ and $Y$. By the axiom schema of separation, $R$ is a set. 
\begin{problem}
    Let $R$ be a binary operation. Prove that $\dom(R),\ran(R)\sub\Un(\Un R)$, hence, by the axiom of union, $\dom(R)$ and $\ran(R)$ are sets.
\end{problem}
\begin{definition}
    Let $R$ be a binary operation on a set $S$, that is, $R\sub X\times X$. We say $R$ is an \hdef{equivalence relation} if the following properties hold.
    \begin{enumerate}
        \item for all $a\in X$, $aRa$;\hfill(reflexive)
        \item $aRb$ implies $bRa$;\hfill(symmetric)
        \item $aRb$ and $bRc$ implies $aRc$.\hfill(transitive)
    \end{enumerate}
    For all $a\in A$, the set ${S}_{a}=\{b\mid aRb\}$ is the \hdef{equivalence class} of $a$. 
\end{definition}
\begin{problem}
    Prove that an equivalence class is a set.
\end{problem}
\begin{example}
    In $\N$, the inclusion $\sub$ is an equivalence relation.
\end{example}
\begin{problem}
    Let $R$ be a binary operation on a set $X$. For all $a,b\in A$, prove that ${S}_{a}\cap{S}_{b}$ is either $\es$ or ${S}_{a}$. Prove that $\Un{S}_{a}=X$, where each pair of ${S}_{a}$ are disjoint.
\end{problem}
\begin{definition}
    Let $\le$ be a binary relation on a set $X$. We say $\le$ is a \hdef{partial ordering} if the following conditions hold:
    \begin{enumerate}
        \item for all $x\in X$, $x\le x$;
        \item for all $x,y\in X$, $x\le y$ and $y\le x$ implies $x=y$;
        \item for all $a,b,c\in X$, if $a\le b$ and $b\le c$, then $a\le c$.
    \end{enumerate}
    The set with a partial ordering is called a \hdef{partially ordered set}.
\end{definition}
\begin{definition}
    A partially ordered set $(X,\le)$ is \hdef{linearly ordered} if for all $p,q\in X$, either $p\le q$ or $q\le p$.
\end{definition}
\begin{example}
    The set of natural numbers $\N$ forms a linearly ordered set in set inclusions.
\end{example}
\begin{proposition}
    Let $(X,\le)$ be a partially ordered set and let $Y\sub X$, then $Y$ is partially ordered.
\end{proposition}
\begin{proof}
    For all elements $a,b,c\in Y$, $a,b,c\in X$, so $Y$ inherits the partial ordering of $X$.
\end{proof}
\begin{problem}
    Let $(X,\le)$ be a partial ordered set, prove that $\le$ is an equivalence relation.
\end{problem}
\begin{problem}
    Let $(X,\le)$ be a linearly ordered set and let $Y\sub X$, prove that $Y$ is linearly ordered.
\end{problem}
\begin{problem}
    Let $X$ be a set. If $(\ps(X),\sub)$ is a linearly ordered set, prove that $X$ is either a singleton or the empty set.
\end{problem}
\begin{definition}
    Let $(X,\le)$ be a partially ordered set and let $Y\sub X$ be an nonempty subset. An element $a$ is the \hdef{upper bound} of $X$ if for all $x\in X$, $x\le a$. An element $b$ is the \hdef{lower bound} of $X$ if for all $x\in X$, $b\le x$. The least upper bound of $X$ is called the \hdef{supremum} and the greatest lower bound of $X$ is called the \hdef{infimum}.
\end{definition}
\begin{theorem}[well-ordering principle]\index{well-ordering principle}
    For all nonempty subset $X\sub\N$, $X$ has a smallest element.
\end{theorem}
\par
The well-ordering principle is equivalent to the axiom of choice, which will be discussed later. You may assume the well-ordering principle is correct for now. 
\begin{theorem}[finite induction]
    Given a subset $S\sub\N$ of the natural numbers with $0\in S$ and $n\in S$ implies $n+1\in S$, then $S=\N$.
\end{theorem}
\begin{proof}
    Suppose $S\ne\N$, then $X=\N\sm S$ is a nonempty set. By the well-ordering principle, $X$ has a smallest element. Since $0\in S$, $0\notin X$, so the minimal element of $X$ can be written in the form $k+1$, where $k\in\N$. Recall that $k+1={k}^{+}=k\cup\{k\}$ is the successor of $k\in\N$, since $k+1$ is the smallest element, $k\notin X$, so $k\in S$. Now we have $k\in S$ and $k+1\notin S$, yet contradiction.
\end{proof}
\begin{example}
    Consider the statement: let $n\in\N$, show that ${\sum}_{i=0}^{n}=(n(n+1))/2$. If $n=0$, then the equation trivially holds. Assume ${\sum}_{i=0}^{k}=(k(k+1))/2$ holds for some $k\in\N$, then ${\sum}_{i=0}^{k+1}=(k(k+1))/2+(k+1)=((k+1)(k+2))/2$. Hence, by induction, ${\sum}_{i=0}^{n}=(n(n+1))/2$ for all $n\in\N$.
\end{example}
\begin{problem}
    Given a subset $S\sub\N$ of the natural numbers with $0\in S$ and $\{0,1,\dots,n\}\sub S$ implies $n+1\in S$, then $S=\N$. This is known as the complete finite induction.
\end{problem}
\begin{problem}
    Prove that the complete finite induction is equivalent to the finite induction.
\end{problem}
\begin{problem}
    Prove that the finite induction implies the well-ordering principle, conclude that they are equivalent.
\end{problem}



\newpage
\section{Functions}
\section{Integers and Cardinality}
\section{Real Numbers}
\section{Algebraic Structures}
\hindex
\end{document}