% chktex-file 1 chktex-file 8 chktex-file 9 chktex-file 12 chktex-file 13 chktex-file 15 chktex-file 17 chktex-file 18 chktex-file 26 chktex-file 31 chktex-file 36 chktex-file 44
\documentclass[10pt]{article}
%%% A Simple Scheme Package, author: Hassium %%% 
%%% version 1.1.1 %%% 

% I quoted some lines from Stackexchange and Github.

% 1. Packages
\usepackage[T1]{fontenc}
\usepackage[hidelinks]{hyperref}
\usepackage[explicit]{titlesec}
\usepackage[utf8]{inputenc}
\usepackage{amsmath,amsthm,amssymb,amsfonts,mathrsfs,mathtools,nicematrix,chngcntr,centernot,ytableau,tikz-cd}
\usepackage{textcomp,tocloft,environ,setspace,geometry,enumerate,enumitem,blindtext,multicol,xcolor,fancyhdr,calligra,graphicx,wrapfig,pgfplots,mdframed,tabularx,lipsum,comment,csquotes}
\usepackage{chemfig}

% 2. General Commands

% Enable useless warnings
% chktex-file 1 
% chktex-file 8
% chktex-file 9
% chktex-file 12
% chktex-file 13
% chktex-file 17
% chktex-file 18
% chktex-file 26
% chktex-file 31
% chktex-file 36
% chktex-file 44

% Page number counter setup
\newcounter{custompage}
\setcounter{custompage}{\numexpr\value{page}+1\relax}
\AddToHook{shipout/before}{\setcounter{custompage}{\numexpr\value{page}+1\relax}}

% Title page setup
\newcommand{\hsetup}{%
    \begin{center}
        \Huge\bfseries\htitle \\
        \vspace{2pt}
        \Large\cfc{\hauthor}
    \end{center} 
    \normalsize
    \vspace{-20.5pt}
    \phantom{}\hrule\phantom{}
}

% Table of Contents
\renewcommand*\contentsname{}
\newcommand{\htoc}{%
    \normalsize
    \begin{multicols}{2}
        \tableofcontents
    \end{multicols}
    \thispagestyle{empty}
    \pagenumbering{gobble}
    \vspace{-6.5pt}\phantom{}\hrule\phantom{}
}

% Page numbering command
\newcommand{\printpage}{\arabic{custompage}}

% Mainmatter setup
\newcommand{\hmain}{%
    \pagestyle{fancy}
    \fancyhead[L]{\cfc{\hauthor}}      
    \fancyhead[R]{\cfd{\printpage}}    
    \fancyhead[C]{\htitle}             
    \renewcommand{\headrulewidth}{0.5pt} 
}

% Adjust the headheight
\setlength{\headheight}{13pt}

% Geometry 
\geometry{letterpaper, margin=0.75in}
\setstretch{1.25}

% Newsection
\makeatletter
\newcommand\newsection[1]{%
  \section*{#1}%
  \addcontentsline{toc}{section}{#1}%
}
\makeatother

% Backslash
\newcommand{\bs}{\backslash}

% Hyperlink on ToC and section titles
\titleformat{\section}
{\normalfont\Large\bfseries}{\thesection}{1em}{\hyperlink{sec-\thesection}{#1}
\addtocontents{toc}{\protect\hypertarget{sec-\thesection}{}}}

% Table of contents section only
\setcounter{tocdepth}{1}

% Description in table of contents
\newcommand{\descr}[1]{%
  \addtocontents{toc}{\medskip\noindent\detokenize{#1}\leavevmode\par\medskip}
}

% Changefont
\newcommand{\cfd}[1]{\fontfamily{pzc}\selectfont{#1}\fontfamily{cmr}\selectfont{}} 
\newcommand{\cfc}[1]{{\usefont{T1}{calligra}{m}{n}#1}} 

% Enumerate
\setlist[enumerate]{topsep=0pt,itemsep=-1ex,partopsep=1ex,parsep=1ex}

% Darkmode
\newcommand{\darkmode}{\pagecolor{black}\color{white}}
\newcommand{\normalmode}{\pagecolor{white}\color{black}}

% Enumerate with tab
\newenvironment{level}
{\addtolength{\itemindent}{2em}}
{\addtolength{\itemindent}{-2em}}

% Circle
\newcommand*\circled[1]{\tikz[baseline=(char.base)]{\node[shape=circle,draw,inner sep=0.5pt](char){#1};}}

% Roman numberals
\newcommand{\rom}{\romannumeral}

% Pgfplot setup
\pgfplotsset{compat=1.18}

% 3. Math

% Theorem styles
\theoremstyle{definition}
\newtheorem{definition}{Definition}[section]
\newtheorem{theorem}{Theorem}[section]
\newtheorem*{proposition}{Proposition}
\newtheorem*{lemma}{Lemma}
\newtheorem*{corollary}{Corollary}
\newtheorem*{example}{Example}
\newtheorem*{remark}{Remark}
\newtheorem*{notation}{Notation}
\newtheorem*{claim}{Claim}
\newtheorem{questioninner}{Exercise}
\newenvironment{exercise}[1][]{%
    \ifx\relax#1\relax\else\renewcommand{\thequestioninner}{#1}\fi 
    \questioninner
}{%
}
\newcommand{\customtheorem}[1]{%
    \expandafter\newtheorem\expandafter*{#1}{#1}%
}

% Remove proofs by empty space
\NewEnviron{reviewmode}{%
    \let\visibleproof\proof
    \let\endvisibleproof\endproof
    \RenewEnviron{proof}{\phantom{}\\\ \\ \\}{}
    \BODY
    \let\proof\visibleproof
    \let\endproof\endvisibleproof
}

% Equation counter
\counterwithin*{equation}{section}
\counterwithin*{equation}{subsection}

% Quiver (Authors: varkor (https://github.com/varkor), AndréC (https://tex.stackexchange.com/users/138900/andr%C3%A9c))
\usetikzlibrary{calc}
\usetikzlibrary{decorations.pathmorphing}
\tikzset{curve/.style={settings={#1},to path={(\tikztostart)
    .. controls ($(\tikztostart)!\pv{pos}!(\tikztotarget)!\pv{height}!270:(\tikztotarget)$)
    and ($(\tikztostart)!1-\pv{pos}!(\tikztotarget)!\pv{height}!270:(\tikztotarget)$)
    .. (\tikztotarget)\tikztonodes}},
    settings/.code={\tikzset{quiver/.cd,#1}
        \def\pv##1{\pgfkeysvalueof{/tikz/quiver/##1}}},
    quiver/.cd,pos/.initial=0.35,height/.initial=0}
\tikzset{tail reversed/.code={\pgfsetarrowsstart{tikzcd to}}}
\tikzset{2tail/.code={\pgfsetarrowsstart{Implies[reversed]}}}
\tikzset{2tail reversed/.code={\pgfsetarrowsstart{Implies}}}
\tikzset{no body/.style={/tikz/dash pattern=on 0 off 1mm}}

% Natural, rational, real, complex numbers, integers
\newcommand{\N}{\mathbb{N}}
\newcommand{\C}{\mathbb{C}}
\newcommand{\R}{\mathbb{R}}
\newcommand{\Q}{\mathbb{Q}}
\newcommand{\Z}{\mathbb{Z}}

% Change font (math) 
\newcommand{\bb}[1]{\mathbb{#1}}
\newcommand{\ca}[1]{\mathcal{#1}}
\newcommand{\fr}[1]{\mathfrak{#1}}

% Topological space
\newcommand{\T}{\mathcal{T}}

% Projective Plane
\newcommand{\Pn}[1]{{\mathbb{P}}^{#1}}
\newcommand{\CP}[1]{{\mathbb{CP}}^{#1}}
\newcommand{\RP}[1]{{\mathbb{RP}}^{#1}}

% Several groups
\DeclareMathOperator{\Sym}{Sym}
\DeclareMathOperator{\GL}{GL}
\DeclareMathOperator{\SL}{SL}
\DeclareMathOperator{\Mod}{Mod}
\newcommand{\Sg}{\mathfrak{S}}
\newcommand{\Ag}{\mathfrak{A}}

% Cayley graph
\DeclareMathOperator{\Cay}{Cay}

% Uniqueness
\newcommand{\uni}{\exists\ \text{!}\ }

% Greek and Hebrew letters
\newcommand{\al}{\alpha}
\newcommand{\be}{\beta}
\newcommand{\ga}{\gamma}
\newcommand{\ep}{\epsilon}
\newcommand{\de}{\delta}
\newcommand{\si}{\sigma}
\newcommand{\la}{\lambda}
\newcommand{\ka}{\kappa}
\newcommand{\om}{\omega}
\newcommand{\vt}{\vartheta}
\newcommand{\vp}{\varphi}
\newcommand{\ve}{\varepsilon}

% Arrows, maps, morphisms
\newcommand{\ua}{\uparrow}
\newcommand{\da}{\downarrow}
\newcommand{\Ra}{\Rightarrow}
\newcommand{\La}{\Leftarrow}
\newcommand{\Ua}{\Uparrow}
\newcommand{\Da}{\Downarrow}
\newcommand{\nRa}{\nRightarrow}
\newcommand{\nLa}{\nLeftarrow}
\newcommand{\hra}{\hookrightarrow}
\newcommand{\hla}{\hookleftarrow}
\newcommand{\lt}{\leadsto}
\newcommand{\mt}{\mapsto}
\newcommand{\rat}{\rightarrowtail}
\newcommand{\lat}{\leftarrowtail}
\newcommand{\thra}{\twoheadrightarrow}
\newcommand{\thla}{\twoheadleftarrow}
\newcommand{\bij}{\xrightarrow{\sim}}

% Overline
\newcommand{\ol}[1]{\overline{#1}}

% Identity
\DeclareMathOperator{\id}{id}

% Sets and inclusions
\newcommand{\sub}{\subset}
\newcommand{\sube}{\subseteq}
\newcommand{\supe}{\supseteq}
\newcommand{\nsub}{\centernot\subset}
\newcommand{\nsup}{\centernot\supset}
\newcommand{\nsube}{\centernot\subseteq}
\newcommand{\nsupe}{\centernot\supseteq}
\newcommand{\subn}{\subsetneq}
\newcommand{\supn}{\supsetneq}
\newcommand{\es}{\varnothing}
\newcommand{\sm}{\setminus}
\newcommand{\ps}{\mathscr{P}}
\newcommand{\Un}{\bigcup}
\newcommand{\In}{\bigcap}
\newcommand{\Du}{\bigsqcup}
\newcommand{\cp}{\amalg}
\newcommand{\Cp}{\coprod}

% Otimes, oplus
\newcommand{\ot}{\otimes}
\newcommand{\op}{\oplus}

% Group action
\newcommand{\acts}{\curvearrowright}

% Span
\DeclareMathOperator{\Span}{span}

% Sign of permutation
\DeclareMathOperator{\sgn}{sgn}

% Normal Subgroup
\newcommand{\nsg}{\trianglelefteq}

% Defined as
\newcommand{\defa}{\coloneqq}

% Semidirect product
\newcommand{\sdp}{\rtimes}

% Inverse
\newcommand{\inv}[1]{{#1}^{-1}}

% Modulo
\renewcommand{\mod}{\ \text{mod}\ }

% Conjugacy classes
\DeclareMathOperator{\Cl}{Cl}

% Holomorph
\DeclareMathOperator{\Hol}{Hol}

% Composition of functions
\newcommand{\comp}{~\raisebox{1pt}{\tikz \draw[line width=0.6pt] circle(1.1pt);}~}

% Galois group
\DeclareMathOperator{\Gal}{Gal}

% Cardinality
\newcommand{\card}[1]{\lvert{#1}\rvert}

% Image
\DeclareMathOperator{\im}{im}

% Norm
\newcommand{\norm}[1]{\lVert{#1}\rVert}

% Partial order
\newcommand{\po}{\preceq}

% Groups generated by
\newcommand{\cyc}[1]{\langle{#1}\rangle}

% Spectrum of a ring
\DeclareMathOperator{\Spec}{Spec}

% Sylow groups
\DeclareMathOperator{\Syl}{Syl}

% Category
\newcommand{\iso}{\approx}
\newcommand{\niso}{\not\approx}
\DeclareMathOperator{\Mor}{Mor}
\DeclareMathOperator{\Aut}{Aut}
\DeclareMathOperator{\End}{End}
\DeclareMathOperator{\Hom}{Hom}
\DeclareMathOperator{\Inn}{Inn}
\DeclareMathOperator{\Out}{Out}
\DeclareMathOperator{\Iso}{Iso}
\DeclareMathOperator{\Ob}{Ob}
\newcommand{\cop}[1]{{#1}^{op}}

% Triangle 
\newcommand{\tri}{\triangle}

% Partial derivative
\newcommand{\pa}{\partial}

% Annihilators
\DeclareMathOperator{\Ann}{Ann}

% 4. Physics & Chemistry

% Quantum: h-bar
\newcommand{\hb}{\hbar}



\endinput
\begin{document}
\def\htitle{An Introduction to Proofs}
\def\hauthor{Hassium}
\hsetup
\htoc
\hmain
\section*{Introduction}\addcontentsline{toc}{section}{Introduction}
In higher-level mathematics, such as algebra, students need ``mathematical maturity'' to understand and apply abstract ideas. There is no obvious way to determine this maturity, nor a clear method to teach someone how to write a proof. This note is designed to serve as a transition to proof-based mathematics, guiding students in adapting to the way mathematics operates.
\par
This introduction is divided into several sections. The first section introduces the basic logic used in proofs. In the second section, we begin to “formalize mathematics” by studying sets and their relations. Functions, which are natural tools for connecting sets while preserving their structure, can be understood as the morphisms between sets, are mentioned in the third section. We shall use facts from these sections in further discussions and renew the idea of them. In the fourth section, we

Integers, denoted $\Z$, are fundamental in our daily lives. Studying integers provides concrete examples for rigorous proofs. Building on the properties of integers, we extend the discussion to infinite sets, exploring questions such as: What is an infinite set? Are these sets “countable”? The final section covers $\R$ and $\C$, the real and complex fields, respectively. It begins with their constructions and presents several algebraic and analytic properties to deepen understanding. This section offers students a first taste of a rigorous mathematics course, so it is highly recommended.
\par
There is no solid prerequisite for this note. I 



\pagebreak

\section{Logic}
Logic is the formal framework and rules of inference that ensure the validity and coherence of arguments in math. 
\begin{remark}
    We shall accept that sentences can be either true or false.
\end{remark} 
A \hdef{proposition} is a sentence that is either true or false in a mathematical system. The label ``true'' or ``false'' assigned to a proposition is called its \hdef{truth value}. We use the letters $T$ and $F$ to represent ``true'' and ``false'', respectively. An \hdef{axiom} is a proposition that is assumed to be true within a mathematical system without requiring proof. Axioms serve as the foundational building blocks of a mathematical theory, from which other propositions can be derived. A \hdef{theorem} is a proposition that has been proven to be true using logical reasoning and the accepted axioms and previously established theorems of the mathematical system. The proof demonstrates why the theorem must hold based on these foundations.
\par
Consider the proposition ``$\pi$ is not a rational number'', which is trivially true. However, we could always find some false companion of this proposition, such as ``$\pi$ is a rational number''. Similarly, we can find a true companion of a false proposition. Let $P$ be a proposition, such companion of $P$ is called the \hdef{negation} of $P$, denoted $\neg P$.
\par
Let $P$ and $Q$ be propositions. Those sentences can be combined using the word ``and'', denoted $P\wedge Q$, and called the \hdef{conjunction} of $P$ and $Q$. The proposition $P\wedge Q$ is true if both $P$ and $Q$ is true. We can combine the propositions by the word ``or'', denoted $P\vee Q$, and called the \hdef{disjunction} of $P$ and $Q$. The proposition $P\vee Q$ is true if at least one of $P$ or $Q$ is true. A \hdef{truth table} is shown below.
\begin{center}
    \begin{tabular}{cc|ccc}
        $P$ & $Q$ & $\neg P$ & $P\wedge Q$ & $P\vee Q$ \\
        \hline
        $T$ & $T$ & $F$ & $T$ & $T$ \\
        $T$ & $F$ & $F$ & $F$ & $T$ \\
        $F$ & $T$ & $T$ & $F$ & $T$ \\
        $F$ & $F$ & $T$ & $F$ & $F$ \\
    \end{tabular}
\end{center}
\par
Two propositions $P$ and $Q$ are \hdef{logically equivalent} if they have the same truth value in every possible combination of truth values for the variables in the statements, denoted $P\equiv Q$.
\begin{example}
    Let $P$, $Q$, and $R$ be propositions. Consider the following statements:
    \begin{enumerate}
        \item $\neg(\neg P)\equiv P$;
        \item $\neg(P\vee Q)\equiv\neg P\wedge\neg Q$;
        \item $\neg(P\wedge Q)\equiv\neg P\vee\neg Q$;
        \item $P\vee(Q\wedge R)\equiv(P\vee Q)\wedge(P\vee R)$.
    \end{enumerate}
    Try to prove or disprove the statements. Based on your results, can you find more properties?
\end{example}
\par
Let $P$ and $Q$ be propositions. Consider the proposition ``if $n$ is a natural number, then $2n$ is an even number''. Let $P$ denotes ``$n$ is a natural number'' and let $Q$ denotes ``$2n$ is an even number'', then the sentence becomes ``if $P$, then $Q$'', denoted $P\implies Q$. This implication called a \hdef{conditional proposition}, $P$ is called the \hdef{antecedent} and $Q$ is called the \hdef{consequent}. The proposition $P\implies Q$ is true if $P$ is true and $Q$ is true. What if $P$ is false? The answer arises from one's intuition.
\begin{example}
    Imagine your high school teacher say ``if you didn't submit your homework, then you haven't completed it''. How would you argue against this sentence? The most likely response would be, ``I did the homework but I didn't submit it''. Whether or not you submitted your homework does not affect the truth value of the implication.
\end{example}
\par
You should convinced by your own intuition (not mine). This case is called a \hdef{vacuous truth}. In the proposition $P\implies Q$, when $P$ is false, $P\implies Q$ is true. The truth table of $P\implies Q$ is shown below.
\begin{center}
    \begin{tabular}{cc|c}
        $P$ & $Q$ & $P\implies Q$ \\
        \hline
        $T$ & $T$ & $T$ \\
        $T$ & $F$ & $F$ \\
        $F$ & $T$ & $T$ \\
        $F$ & $F$ & $T$
    \end{tabular}
\end{center}
\par
Let $P$ and $Q$ be propositions, $(P\implies Q)\wedge(Q\implies P)$ is called a \hdef{biconditional proposition}, denoted $P\iff Q$. We will write this by ``$P$ is true if and only if $Q$ is true''.
\begin{example}
    Let $P$ and $Q$ be propositions. Try to find a proposition $R$ by ``$\vee$'', ``$\wedge$'', and ``$\neg$'' such that $R\equiv P\implies Q$.
\end{example}
\begin{example}
    Write down the truth table of a biconditional proposition. Based on your truth table and last example, try to find a proposition $R$ by ``$\vee$'', ``$\wedge$'', and ``$\neg$'' such that $R\equiv P\iff Q$. If $P\iff Q$ is true, does $P\equiv Q$?
\end{example}
\begin{example}
    Let $P$, $Q$, and $R$ be propositions. Try to prove or disprove $P\implies(Q\vee R)\equiv\neg P\vee Q\vee R$.
\end{example}
\par
Given a proposition $P\implies Q$, the \hdef{converse} is defined as $Q\implies P$ and the \hdef{contrapositive} is defined as $\neg Q\implies\neg P$. The truth table is shown below, and it suffices to conclude that $P\implies Q\equiv\neg Q\implies\neg P$.
\begin{center}
    \begin{tabular}{cc|ccc}
        $P$ & $Q$ & $P\implies Q$ & $Q\implies P$ & $\neg Q\implies\neg P$ \\
        \hline
        $T$ & $T$ & $T$ & $T$ & $T$ \\
        $T$ & $F$ & $F$ & $T$ & $F$ \\
        $F$ & $T$ & $T$ & $F$ & $T$ \\
        $F$ & $F$ & $T$ & $T$ & $T$
    \end{tabular}
\end{center}
\par
Let $P$ be the proposition ``$x$ is a natural number''. Here $x$ is a \hdef{variable}, and the truth value of this proposition depends on $x$. For instance, if $x=1$, then $P$ is true; if $x=0.86$, then $P$ is false. A \hdef{propositional function} is a family of propositions depending on one or more variables. The collection of permitted variables is the \hdef{domain}. Now we write $P(x)$ instead of $P$, so $P(1)$ is true and $P(0.86)$ is false.
\par
Propositional functions are often quantified. The \hdef{universal quantifier} is denoted by ``$\forall$'', and the proposition $\forall x(P(x))$ is true if and only if $P(x)$ is true for every $x$ in its domain. The \hdef{existential quantifier} is denoted by ``$\exists$'', and the proposition $\exists x(P(x))$ is true if and only if $P(x)$ is true for at least one $x$ in its domain.
\par
Consider the proposition $\forall x(P(x))$, this means all $x$ make $P(x)$ true, so there does not exists some $x$ such that $P(x)$ is false, which is $\neg(\exists x(\neg P(x)))$. 
\begin{example}
    Let $P(x)$ be a proposition. Consider the following statements:
    \begin{enumerate}
        \item $\neg(\forall x(P(x)))\iff\exists x(\neg P(x))$;
        \item $\neg(\exists x(P(x)))\iff\forall x(\neg P(x))$.
    \end{enumerate}
    Try to prove or disprove the statements.
\end{example}
\par
In the following sections, we shall assume readers are familiar with basic logic. Several expressions and their “translations” are shown below.
\begin{center}
    \begin{tabular}{c|c}
        $P\implies Q$ & $P\iff Q$ \\
        \hline
        $P$ implies $Q$; if $P$, then $Q$ & $P$ if and only if $Q$ \\
        $P$ is sufficient for $Q$; $Q$ is necessary for $P$ & $P$ is necessary and sufficient for $Q$
    \end{tabular}
\end{center}
\begin{example}
    Here are some true propositions, try to convert them into the language of logic.
    \begin{enumerate}
        \item If $x$ is a natural number, then $2x$ is an even number.
        \item For all natural number $x$, there exists a natural number $y$ such that $y>x$.
        \item A sequence $({x}_{n})$ of real numbers is a \hdef{Cauchy sequence} if for every positive real number $\ep$, there exists some $N\in\N$ and $N\ne 0$ such that for all natural numbers $m,n>N$, $\card{{x}_{m}-{x}_{n}}<\ep$.
    \end{enumerate}
    For every statement above, you don't need to understand the contents, try to understand the logical relations.
\end{example}
\begin{example}
    Based on your results, try to find the negation and converse of those propositions.
\end{example}
\section{Sets}
\par
In this section, we begin to investigate sets, the most basic entities in mathematics. It is natural to ask the question: what is a set? There is no precise definition of sets. Intuitively, a \hdef{set} is a collection of objects, and those objects are called \hdef{elements}. From now on, we shall accept that sets exist.
\par
If $S$ is a set and $x$ is an element in $S$, then we say $x$ belongs to $S$, denoted $x\in S$. If $x$ does not belong to $S$, then we write $x\notin S$. If $S$ has no element, then we call it an \hdef{empty set}, denoted $\es$.
\par
One way to describe a set is to explicitly list the elements. For instance, we can write a set $S=\{6,7,8\}$. Another way is to express the elements by some properties they satisfied.
\begin{example}
    The set of rational numbers is the set $\Q=\{a/b\mid a,b\in\Z\ \text{and}\ b\ne 0\}$, where $\Z$ is the set of integers.
\end{example}
\begin{example}
    The set $\{2n\mid n\in\N\}$ is the set of all even numbers. Try to write the set of all odd numbers.
\end{example}
\begin{definition}
    Let $S$ be a set. A set $R$ is a \hdef{subset} of $S$, denoted $R\sub S$, if for all $x\in R$, $x\in S$. If there exists some $x\in S$ such that $x\notin R$, then $R$ is called a \hdef{proper subset} of $S$, denoted $R\subn S$.
\end{definition}
\begin{remark}
    Some textbooks use ``$\sube$'' for subsets and ``$\sub$'' for proper subsets.
\end{remark}
\begin{example}
    For all sets $A$, we have $A\sub A$.
\end{example}
\begin{proposition}
    Let $X$ and $Y$ be sets, then $X=Y$ if and only $X\sub Y$ and $Y\sub X$.
\end{proposition}
\begin{remark}
    For a biconditional proposition $P\iff Q$, we use the notation ``$(\Ra)$'' in the proof to show $P\implies Q$ and ``$(\La)$'' for $Q\implies P$.
\end{remark}
\begin{proof}
    Let $X$ and $Y$ be sets. $(\Ra)$ For all $x\in X$, since $X=Y$, $x\in Y$, so $X\sub Y$. For all $y\in Y$, since $X=Y$, $y\in X$, so $Y\sub X$. $(\La)$ Suppose $X\ne Y$, then there exists $a\in X$ and $a\notin Y$, so $X\nsub Y$, yet contradiction. 
\end{proof}
\begin{proposition}
    Let $A$ be any set, then $\es\sub A$.
\end{proposition}
\begin{proof}
    Suppose $\es\nsub A$, then there exists $x\in\es$ such that $x\notin A$, since $x\in\es$ is false, contradiction.
\end{proof}
\begin{example}
    If $X$, $Y$, and $Z$ are sets such that $X\sub Y$ and $Y\sub Z$, then $X\sub Z$.
\end{example}
\begin{definition}
    Let $A$ and $B$ be sets. The \hdef{union} of $A$ and $B$ is the set $\{x\mid x\in A\ \text{or}\ x\in B\}$, denoted $A\cup B$. The \hdef{intersection} of $A$ and $B$ is the set $\{x\mid x\in A\ \text{and}\ x\in B\}$. The \hdef{complement} of $A$ in $B$ is the set $\{x\mid x\in B\ \text{and}\ x\notin A\}$, denoted $B/A$.
\end{definition}
\par
Some textbooks assume there exists a ``universal set'', denoted $U$, which has all objects as elements including itself, so we can define complements of any set $S$ as the set $U/S$. However, this assumption leads to a paradox. 
\begin{example}
    Consider the set $S$, defined as the set of all sets that are not members of themselves, that is, $S=\{X\mid X\notin X\}$. Does $S$ belong to $S$? This is known as the \hdef{Russell's Paradox}.
\end{example}
\par
Assume $S\in S$. By the definition of $S$, it must satisfy $S\notin S$. This is a contradiction. Assume $S \notin S$. By the definition of $S$, it must satisfy $S\in S$. This is also a contradiction. Thus, the existence of such a set $S$ leads to a logical inconsistency. 








\newpage
\section{Functions}
\section{Integers}
\section{Cardinality}
\section{Real and Complex Numbers}



\newsection{Acknowledgment}

 





\hindex
\end{document}