%%% A Simple Format, author: Hassium %%% 
%%% version 1.2.4 %%% 

% I quoted some lines from Stackexchange and Github.

%%%%%%%%%%%%%%%%%%%%%
%%%     Cursor    %%% 
%%%    Parking    %%%    
%%%      Lot      %%%
%%%%%%%%%%%%%%%%%%%%%

% 1. Packages
\usepackage[T1]{fontenc}
\usepackage[explicit]{titlesec}
\usepackage[utf8]{inputenc}
\usepackage{amsmath,amsthm,amssymb,amsfonts,mathrsfs,mathtools,nicematrix,chngcntr,centernot,ytableau,tikz-cd}
\usepackage{imakeidx,textcomp,tocloft,environ,setspace,geometry,enumerate,enumitem,blindtext,multicol,xcolor,fancyhdr,calligra,graphicx,wrapfig,pgfplots,mdframed,tabularx,lipsum,comment,csquotes,verbatim,transparent,scalerel,ragged2e,halloweenmath}
\usepackage[hidelinks]{hyperref}
\usepackage{chemfig}

% 2. General Commands

% chktex-file 1 chktex-file 8 chktex-file 9 chktex-file 12 chktex-file 13 chktex-file 14 chktex-file 15 chktex-file 17 chktex-file 18 chktex-file 26 chktex-file 31 chktex-file 36 chktex-file 40 chktex-file 44

% Title page
\newcommand{\hsetup}{%
    \thispagestyle{empty}
    \begin{center}
        \LARGE\bfseries\htitle \\
        \vspace{2pt}
        \small\texttt{\hauthor}
    \end{center} 
    \normalsize
}

% Mainmatter: fancy header
\newcommand{\hmain}{%
    \pagestyle{fancy}
    \fancyhead[L]{\texttt{\hfauthor}}      
    \fancyhead[R]{\texttt{p.\thepage}}
    \fancyhead[C]{\texttt{\htitle}}             
    \renewcommand{\headrulewidth}{0.5pt}
    \fancyfoot{}
}

% Page Geometry 
\geometry{letterpaper,top=54pt,
    bottom=46.8pt,
    marginparsep=5.67pt,
    marginparwidth=56.69pt,
    voffset=0pt,
    hoffset=0pt,
    left=54pt,
    right=54pt,
    headheight=24pt,
    headsep=10pt
}
\setstretch{1.25}

% Backslash
\newcommand{\bs}{\backslash}

% Changem fonts
\newcommand{\cfd}[1]{\fontfamily{pzc}\selectfont #1\fontfamily{cmr}\selectfont}
\newcommand{\cfc}[1]{{\usefont{T1}{calligra}{m}{n}#1}} 
\newcommand{\cft}[1]{\texttt{#1}}

% Enumerate
\setlist[enumerate]{topsep=0pt,itemsep=-1ex,partopsep=1ex,parsep=1ex}

% Darkmode
\newcommand{\darkhsetup}{%
    \thispagestyle{empty}
    \begin{center}
        {\LARGE\bfseries\textcolor{white}{\htitle}} \\
        \vspace{2pt}
        {\small\texttt{\textcolor{white}{\hauthor}}}
    \end{center} 
    \normalsize
}

\newcommand{\darkhmain}{
    \pagestyle{fancy}
    \pagecolor{black}
    \color{white}     
    \fancyhead[L]{\textcolor{white}{\texttt{\hfauthor}}}  
    \fancyhead[R]{\textcolor{white}{\texttt{p.\thepage}}}
    \fancyhead[C]{\textcolor{white}{\texttt{\htitle}}  }           
    \renewcommand{\headrule}{\color{white}\hrule width\headwidth height0.5pt}
    \fancyfoot{}
}

% Enumerate with tab
\newenvironment{level}{
    \addtolength{\itemindent}{2em}}{
    \addtolength{\itemindent}{-2em}
}

% Circle
\newcommand*\circled[1]{\tikz[baseline=(char.base)]{\node[shape=circle,draw,inner sep=0.5pt](char){#1};}}

% Roman numberals
\newcommand{\rom}{\romannumeral}

% Diamond
\newcommand{\largediamond}{\scalerel*{\lozenge}{\square}}

% Pgfplot setup
\pgfplotsset{compat=1.18}

% 3. Math

% Theorem styles
\theoremstyle{definition}
\newtheorem{definition}{Definition}[section]
\newtheorem{theorem}{Theorem}[section]
\newtheorem*{proposition}{Proposition}
\newtheorem*{lemma}{Lemma}
\newtheorem*{corollary}{Corollary}
\newtheorem*{example}{Example}
\newtheorem*{remark}{Remark}
\newtheorem*{notation}{Notation}
\newtheorem{problem}{Problem}[section]
\newtheorem*{claim}{Claim}
\newtheorem*{answer}{Answer}
\newtheorem{questioninner}{Exercise}
\newenvironment{exercise}[1][]{%
    \ifx\relax#1\relax\else\renewcommand{\thequestioninner}{#1}\fi 
    \questioninner
}{%
}
\newcommand{\customtheorem}[1]{%
    \expandafter\newtheorem\expandafter*{#1}{#1}%
}
\newenvironment{hdefinition}[1][]{%
    \ifx\relax#1\relax
        \begin{definition}
    \else
        \begin{definition}[#1]\index{#1}
    \fi
    \setlength{\baselineskip}{12pt}
}{%
    \end{definition}
}
\newenvironment{htheorem}[1][]{%
    \ifx\relax#1\relax
        \begin{theorem}
    \else
        \begin{theorem}[#1]\index{#1}
    \fi
    \setlength{\baselineskip}{12pt}
}{%
    \end{theorem}
}

% Remove proofs by empty space
\NewEnviron{reviewmode}{%
    \let\visibleproof\proof
    \let\endvisibleproof\endproof
    \RenewEnviron{proof}{\phantom{}\\\ \\ \\}{}
    \BODY
    \let\proof\visibleproof
    \let\endproof\endvisibleproof
}

% Equation counter
\counterwithin*{equation}{section}
\counterwithin*{equation}{subsection}

% Quiver (Authors: varkor (https://github.com/varkor), AndréC (https://tex.stackexchange.com/users/138900/andr%C3%A9c))
\usetikzlibrary{calc}
\usetikzlibrary{decorations.pathmorphing}
\tikzset{curve/.style={settings={#1},to path={(\tikztostart)
    .. controls ($(\tikztostart)!\pv{pos}!(\tikztotarget)!\pv{height}!270:(\tikztotarget)$)
    and ($(\tikztostart)!1-\pv{pos}!(\tikztotarget)!\pv{height}!270:(\tikztotarget)$)
    .. (\tikztotarget)\tikztonodes}},
    settings/.code={\tikzset{quiver/.cd,#1}
        \def\pv##1{\pgfkeysvalueof{/tikz/quiver/##1}}},
    quiver/.cd,pos/.initial=0.35,height/.initial=0}
\tikzset{tail reversed/.code={\pgfsetarrowsstart{tikzcd to}}}
\tikzset{2tail/.code={\pgfsetarrowsstart{Implies[reversed]}}}
\tikzset{2tail reversed/.code={\pgfsetarrowsstart{Implies}}}
\tikzset{no body/.style={/tikz/dash pattern=on 0 off 1mm}}

% Natural, rational, real, complex numbers, integers, fields
\newcommand{\N}{\mathbb{N}}
\newcommand{\C}{\mathbb{C}}
\newcommand{\R}{\mathbb{R}}
\newcommand{\Q}{\mathbb{Q}}
\newcommand{\Z}{\mathbb{Z}}
\newcommand{\F}{\mathbb{F}}

% Change font (math) 
\newcommand{\bb}[1]{\mathbb{#1}}
\newcommand{\ca}[1]{\mathcal{#1}}
\newcommand{\fr}[1]{\mathfrak{#1}}

% Topological space
\newcommand{\T}{\mathcal{T}}

% Projective space
\newcommand{\Ps}[1]{{\mathbb{P}}^{#1}}
\newcommand{\CP}[1]{{\mathbb{CP}}^{#1}}
\newcommand{\RP}[1]{{\mathbb{RP}}^{#1}}

% Several groups
\DeclareMathOperator{\Sym}{Sym}
\DeclareMathOperator{\GL}{GL}
\DeclareMathOperator{\SL}{SL}
\DeclareMathOperator{\Mod}{Mod}
\newcommand{\Sg}{\mathfrak{S}}
\newcommand{\Ag}{\mathfrak{A}}

% Cayley graph
\DeclareMathOperator{\Cay}{Cay}

% There exists a unique...
\newcommand{\uni}{\exists\ \text{!}\ }

% Greek and Hebrew letters
\newcommand{\al}{\alpha}
\newcommand{\be}{\beta}
\newcommand{\ga}{\gamma}
\newcommand{\ep}{\epsilon}
\newcommand{\de}{\delta}
\newcommand{\si}{\sigma}
\newcommand{\la}{\lambda}
\newcommand{\ka}{\kappa}
\newcommand{\om}{\omega}
\newcommand{\Ga}{\Gamma}
\newcommand{\De}{\Delta}
\newcommand{\Si}{\Sigma}
\newcommand{\LA}{\Lambda}
\newcommand{\Om}{\Omega}
\newcommand{\vt}{\vartheta}
\newcommand{\vp}{\varphi}
\newcommand{\ve}{\varepsilon}

% Arrows, maps, morphisms
\newcommand{\ua}{\uparrow}
\newcommand{\da}{\downarrow}
\newcommand{\Ra}{\Rightarrow}
\newcommand{\La}{\Leftarrow}
\newcommand{\Ua}{\Uparrow}
\newcommand{\Da}{\Downarrow}
\newcommand{\nRa}{\nRightarrow}
\newcommand{\nLa}{\nLeftarrow}
\newcommand{\hra}{\hookrightarrow}
\newcommand{\hla}{\hookleftarrow}
\newcommand{\lt}{\leadsto}
\newcommand{\mt}{\mapsto}
\newcommand{\rat}{\rightarrowtail}
\newcommand{\lat}{\leftarrowtail}
\newcommand{\thra}{\twoheadrightarrow}
\newcommand{\thla}{\twoheadleftarrow}
\newcommand{\bij}{\xrightarrow{\sim}}

% Wide bar (by Bryce)
\makeatletter
\let\save@mathaccent\mathaccent
\newcommand*\if@single[3]{%
 \setbox0\hbox{${\mathaccent"0362{#1}}^H$}%
 \setbox2\hbox{${\mathaccent"0362{\kern0pt#1}}^H$}%
 \ifdim\ht0=\ht2 #3\else #2\fi
 }
%The bar will be moved to the right by a half of \macc@kerna, which is computed by amsmath:
\newcommand*\rel@kern[1]{\kern#1\dimexpr\macc@kerna}
%If there's a superscript following the bar, then no negative kern may follow the bar;
%an additional {} makes sure that the superscript is high enough in this case:
\newcommand*\wb[1]{\@ifnextchar^{{\wide@bar{#1}{0}}}{\wide@bar{#1}{1}}}
%Use a separate algorithm for single symbols:
\newcommand*\wide@bar[2]{\if@single{#1}{\wide@bar@{#1}{#2}{1}}{\wide@bar@{#1}{#2}{2}}}
\newcommand*\wide@bar@[3]{%
 \begingroup
 \def\mathaccent##1##2{%
%Enable nesting of accents:
 \let\mathaccent\save@mathaccent
%If there's more than a single symbol, use the first character instead (see below):
 \if#32 \let\macc@nucleus\first@char \fi
%Determine the italic correction:
 \setbox\z@\hbox{$\macc@style{\macc@nucleus}_{}$}%
 \setbox\tw@\hbox{$\macc@style{\macc@nucleus}{}_{}$}%
 \dimen@\wd\tw@
 \advance\dimen@-\wd\z@
%Now \dimen@ is the italic correction of the symbol.
 \divide\dimen@ 3
 \@tempdima\wd\tw@
 \advance\@tempdima-\scriptspace
%Now \@tempdima is the width of the symbol.
 \divide\@tempdima 10
 \advance\dimen@-\@tempdima
%Now \dimen@ = (italic correction / 3) - (Breite / 10)
 \ifdim\dimen@>\z@ \dimen@0pt\fi
%The bar will be shortened in the case \dimen@<0 !
 \rel@kern{0.6}\kern-\dimen@
 \if#31
 \overline{\rel@kern{-0.6}\kern\dimen@\macc@nucleus\rel@kern{0.4}\kern\dimen@}%
 \advance\dimen@0.4\dimexpr\macc@kerna
%Place the combined final kern (-\dimen@) if it is >0 or if a superscript follows:
 \let\final@kern#2%
 \ifdim\dimen@<\z@ \let\final@kern1\fi
 \if\final@kern1 \kern-\dimen@\fi
 \else
 \overline{\rel@kern{-0.6}\kern\dimen@#1}%
 \fi
 }%
 \macc@depth\@ne
 \let\math@bgroup\@empty \let\math@egroup\macc@set@skewchar
 \mathsurround\z@ \frozen@everymath{\mathgroup\macc@group\relax}%
 \macc@set@skewchar\relax
 \let\mathaccentV\macc@nested@a
%The following initialises \macc@kerna and calls \mathaccent:
 \if#31
 \macc@nested@a\relax111{#1}%
 \else
%If the argument consists of more than one symbol, and if the first token is
%a letter, use that letter for the computations:
 \def\gobble@till@marker##1\endmarker{}%
 \futurelet\first@char\gobble@till@marker#1\endmarker
 \ifcat\noexpand\first@char A\else
 \def\first@char{}%
 \fi
 \macc@nested@a\relax111{\first@char}%
 \fi
 \endgroup
}
\makeatother

% Identity
\DeclareMathOperator{\id}{id}

% Sets and inclusions
\newcommand{\sub}{\subset}
\newcommand{\sube}{\subseteq}
\newcommand{\sups}{\supset}
\newcommand{\supe}{\supseteq}
\newcommand{\nsub}{\centernot\subset}
\newcommand{\nsup}{\centernot\supset}
\newcommand{\nsube}{\centernot\subseteq}
\newcommand{\nsupe}{\centernot\supseteq}
\newcommand{\subn}{\subsetneq}
\newcommand{\supn}{\supsetneq}
\newcommand{\es}{\varnothing}
\newcommand{\sm}{\setminus}
\newcommand{\ps}{\mathscr{P}}
\newcommand{\Un}{\bigcup}
\newcommand{\In}{\bigcap}
\newcommand{\Du}{\bigsqcup}
\newcommand{\cp}{\amalg}
\newcommand{\Cp}{\coprod}

% Height
\DeclareMathOperator{\Ht}{ht}

% Abelianization
\DeclareMathOperator{\ab}{ab}

% Otimes, oplus
\newcommand{\ot}{\otimes}
\newcommand{\op}{\oplus}

% Group action
\newcommand{\acts}{\curvearrowright}

% Sign of permutation
\DeclareMathOperator{\sgn}{sgn}

% Normal Subgroup
\newcommand{\nsg}{\trianglelefteq}

% Defined as
\newcommand{\defa}{\coloneqq}

% Semidirect product
\newcommand{\sdp}{\rtimes}

% Inverse
\newcommand{\inv}[1]{{#1}^{-1}}

% Modulo
\renewcommand{\mod}{\ \text{mod}\ }

% Conjugacy classes
\DeclareMathOperator{\Cl}{Cl}

% Holomorph
\DeclareMathOperator{\Hol}{Hol}

% Composition of functions
\newcommand{\comp}{~\raisebox{1pt}{\tikz \draw[line width=0.6pt] circle(1.1pt);}~}

% Galois group
\DeclareMathOperator{\Gal}{Gal}

% Cardinality
\newcommand{\card}[1]{\lvert{#1}\rvert}

% Image
\DeclareMathOperator{\im}{im}

% Norm
\newcommand{\norm}[1]{\lVert{#1}\rVert}

% Partial order
\newcommand{\po}{\prec}
\newcommand{\poe}{\preceq}

% Groups generated by
\newcommand{\cyc}[1]{\langle{#1}\rangle}

% Spectrum of a ring
\DeclareMathOperator{\Spec}{Spec}

% Sylow groups
\DeclareMathOperator{\Syl}{Syl}

% Category
\newcommand{\iso}{\approx}
\newcommand{\niso}{\not\approx}
\DeclareMathOperator{\Mor}{Mor}
\DeclareMathOperator{\Aut}{Aut}
\DeclareMathOperator{\End}{End}
\DeclareMathOperator{\Hom}{Hom}
\DeclareMathOperator{\Inn}{Inn}
\DeclareMathOperator{\Out}{Out}
\DeclareMathOperator{\Iso}{Iso}
\DeclareMathOperator{\Ob}{Ob}
\DeclareMathOperator{\dom}{dom}
\DeclareMathOperator{\cod}{cod}
\newcommand{\cop}[1]{{\texttt{#1}}^{\texttt{{op}}}}
\newcommand{\cat}[1]{\cft{#1}}

% Floor
\newcommand{\fl}[1]{\lfloor{#1}\rfloor}

% Square
\newcommand{\sq}{\square}

% Range
\DeclareMathOperator{\ran}{ran}

% Triangle 
\newcommand{\tri}{\triangle}

% Partial derivative
\newcommand{\pa}{\partial}

% Annihilators
\DeclareMathOperator{\Ann}{Ann}

% Affine n-space
\newcommand{\A}[1]{{\bb{A}}^{#1}}

% CAT space
\DeclareMathOperator{\CAT}{CAT}



\endinput