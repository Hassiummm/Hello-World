% chktex-file 1 chktex-file 8 chktex-file 9 chktex-file 12 chktex-file 13 chktex-file 15 chktex-file 17 chktex-file 18 chktex-file 26 chktex-file 31 chktex-file 36 chktex-file 44
\documentclass[10pt]{article}
%%% This is a scheme of a simple package. %%%
% version 1.0.6

% 1. Packages
\usepackage[T1]{fontenc}
\usepackage[hidelinks]{hyperref}
\usepackage[explicit]{titlesec}
\usepackage[utf8]{inputenc}
\usepackage{amsmath,amsthm,amssymb,amsfonts,mathrsfs,mathtools,nicematrix,chngcntr,centernot,ytableau,tikz-cd}
\usepackage{environ,setspace,geometry,enumerate,enumitem,blindtext,multicol,xcolor,calligra,graphicx,wrapfig,pgfplots,mdframed,tabularx,lipsum,comment,csquotes}
\usepackage{chemfig}

% 2. General Commands

% Enable useless warnings
% chktex-file 1 
% chktex-file 36 
% chktex-file 12 
% chktex-file 26 
% chktex-file 18

% Multicolumn setup
\setlength{\columnseprule}{1pt}
\def\columnseprulecolor{\color{black}}

% Geometry 
\geometry{letterpaper, margin=0.75in}
\setstretch{1.25}

% Newsection (shown in ToC, no counter)
\makeatletter
\newcommand\newsection[1]{%
  \section*{#1}%
  \addcontentsline{toc}{section}{#1}%
}
\makeatother

% Backslash
\newcommand{\bs}{\backslash}

% Hyperlink on ToC and section titles
\titleformat{\section}
{\normalfont\Large\bfseries}{\thesection}{1em}{\hyperlink{sec-\thesection}{#1}
\addtocontents{toc}{\protect\hypertarget{sec-\thesection}{}}}
\titleformat{name=\section,numberless}
{\normalfont\Large\bfseries}{}{0pt}{\hyperlink{toc}{#1}}

% Table of contents section only
\setcounter{tocdepth}{1}

% Changefont
\newcommand{\cfd}[1]{\fontfamily{pzc}\selectfont{#1}\fontfamily{cmr}\selectfont{}} 
\newcommand{\cfc}[1]{\calligra{#1}\fontfamily{cmr}\selectfont{}} 

% Enumerate
\setlist[enumerate]{topsep=0pt,itemsep=-1ex,partopsep=1ex,parsep=1ex}

% Remove proofs by empty space
\NewEnviron{reviewmode}{%
    \let\visibleproof\proof
    \let\endvisibleproof\endproof
    \RenewEnviron{proof}{\phantom{}\\\ \\ \\}{}
    \BODY
    \let\proof\visibleproof
    \let\endproof\endvisibleproof
}

% Darkmode (black background, white text)
\newcommand{\darkmode}{\pagecolor{black}\color{white}}

% Enumerate with tab
\newenvironment{level}
{\addtolength{\itemindent}{2em}}
{\addtolength{\itemindent}{-2em}}

% Circle
\newcommand*\circled[1]{\tikz[baseline=(char.base)]{\node[shape=circle,draw,inner sep=0.5pt](char){#1};}}

% Roman numberals
\newcommand{\rom}{\romannumeral}

% Pgfplot setup
\pgfplotsset{compat=1.18}

% 3. Math

% Theorem styles
\theoremstyle{definition}
\newtheorem{definition}{Definition}[section]
\newtheorem{theorem}{Theorem}[section]
\newtheorem*{proposition}{Proposition}
\newtheorem*{lemma}{Lemma}
\newtheorem*{corollary}{Corollary}
\newtheorem*{example}{Example}
\newtheorem*{remark}{Remark}
\newtheorem*{notation}{Notation}
\newtheorem{questioninner}{Exercise}
\newenvironment{exercise}[1][]{%
    \ifx\relax#1\relax\else\renewcommand{\thequestioninner}{#1}\fi 
    \questioninner
}{%
}
\makeatletter % title: exercise, with customize []
\newenvironment{x}[1]{%
    \par\medskip\noindent\textbf{#1\@addpunct{.}}\hskip\labelsep
}{%
    \par 
}
\makeatother % general environment for any title you want

% Equation counter
\counterwithin*{equation}{section}
\counterwithin*{equation}{subsection}

% Quiver (Authors: varkor (https://github.com/varkor), AndréC (https://tex.stackexchange.com/users/138900/andr%C3%A9c))
\usetikzlibrary{calc}
\usetikzlibrary{decorations.pathmorphing}
\tikzset{curve/.style={settings={#1},to path={(\tikztostart)
    .. controls ($(\tikztostart)!\pv{pos}!(\tikztotarget)!\pv{height}!270:(\tikztotarget)$)
    and ($(\tikztostart)!1-\pv{pos}!(\tikztotarget)!\pv{height}!270:(\tikztotarget)$)
    .. (\tikztotarget)\tikztonodes}},
    settings/.code={\tikzset{quiver/.cd,#1}
        \def\pv##1{\pgfkeysvalueof{/tikz/quiver/##1}}},
    quiver/.cd,pos/.initial=0.35,height/.initial=0}
\tikzset{tail reversed/.code={\pgfsetarrowsstart{tikzcd to}}}
\tikzset{2tail/.code={\pgfsetarrowsstart{Implies[reversed]}}}
\tikzset{2tail reversed/.code={\pgfsetarrowsstart{Implies}}}
\tikzset{no body/.style={/tikz/dash pattern=on 0 off 1mm}}

% Change font (math) 
\newcommand{\bb}[1]{\mathbb{#1}}
\newcommand{\ca}[1]{\mathcal{#1}}
\newcommand{\fr}[1]{\mathfrak{#1}}

% Natural, rational, real, complex numbers, integers
\newcommand{\N}{\mathbb{N}}
\newcommand{\C}{\mathbb{C}}
\newcommand{\R}{\mathbb{R}}
\newcommand{\Q}{\mathbb{Q}}
\newcommand{\Z}{\mathbb{Z}}

% Topological space
\newcommand{\T}{\mathcal{T}}

% Fancy symmetric group and alternating group
\newcommand{\Sg}{\mathfrak{S}}
\newcommand{\Ag}{\mathfrak{A}}

% Symmetric group
\DeclareMathOperator{\Sym}{Sym}

% Uniqueness
\newcommand{\uni}{\exists\ \text{!}\ }

% Greek and Hebrew letters
\newcommand{\al}{\alpha}
\newcommand{\be}{\beta}
\newcommand{\ga}{\gamma}
\newcommand{\ep}{\epsilon}
\newcommand{\de}{\delta}
\newcommand{\si}{\sigma}
\newcommand{\la}{\lambda}
\newcommand{\ka}{\kappa}
\newcommand{\vt}{\vartheta}
\newcommand{\vp}{\varphi}
\newcommand{\ve}{\varepsilon}

% Arrows, maps, morphisms
\newcommand{\ua}{\uparrow}
\newcommand{\da}{\downarrow}
\newcommand{\Ra}{\Rightarrow}
\newcommand{\La}{\Leftarrow}
\newcommand{\Ua}{\Uparrow}
\newcommand{\Da}{\Downarrow}
\newcommand{\nRa}{\nRightarrow}
\newcommand{\nLa}{\nLeftarrow}
\newcommand{\hra}{\hookrightarrow}
\newcommand{\hla}{\hookleftarrow}
\newcommand{\lt}{\leadsto}
\newcommand{\mt}{\mapsto}
\newcommand{\rat}{\rightarrowtail}
\newcommand{\lat}{\leftarrowtail}
\DeclareMathOperator{\id}{id}
\newcommand{\bij}{\xrightarrow{\sim}} % right arrow with \sim on it

% Sets, inclusions
\newcommand{\sub}{\subset}
\newcommand{\sube}{\subseteq}
\newcommand{\supe}{\supseteq}
\newcommand{\nsub}{\centernot\subset}
\newcommand{\nsup}{\centernot\supset}
\newcommand{\nsube}{\centernot\subseteq}
\newcommand{\nsupe}{\centernot\supseteq}
\newcommand{\es}{\varnothing}
\newcommand{\sm}{\setminus}
\newcommand{\ps}{\mathscr{P}}

% Otimes, oplus
\newcommand{\ot}{\otimes}
\newcommand{\op}{\oplus}

% Sign of permutation
\DeclareMathOperator{\sgn}{sgn}

% Normal Subgroup
\newcommand{\nsg}{\trianglelefteq}

% Defined as
\newcommand{\defa}{\coloneqq}

% Semidirect product
\newcommand{\sdp}{\rtimes}

% Inverse
\newcommand{\inv}[1]{{#1}^{-1}}

% Union, intersection, disjoint union
\newcommand{\Cap}{\bigcap}
\newcommand{\Cup}{\bigcup}
\newcommand{\DU}{\bigsqcup}

% Modulo
\renewcommand{\mod}{\ \text{mod}\ }

% Conjugacy classes
\DeclareMathOperator{\Cl}{Cl}

% Holomorph
\DeclareMathOperator{\Hol}{Hol}

% Composition of functions
\newcommand{\comp}{~\raisebox{1pt}{\tikz \draw[line width=0.6pt] circle(1.1pt);}~}

% Galois group
\DeclareMathOperator{\gal}{Gal}

% Cardinality
\newcommand{\card}[1]{\lvert{#1}\rvert}

% Norm
\newcommand{\norm}[1]{\lVert{#1}\rVert}

% Partial order
\newcommand{\po}{\preceq}

% Groups generated by
\newcommand{\cyc}[1]{\langle{#1}\rangle}

% Spectrum of a ring
\DeclareMathOperator{\Spec}{Spec}

% Sylow groups
\DeclareMathOperator{\Syl}{Syl}

% Category
\newcommand{\iso}{\approx}
\DeclareMathOperator{\Aut}{Aut}
\DeclareMathOperator{\End}{End}
\DeclareMathOperator{\Hom}{Hom}
\DeclareMathOperator{\Inn}{Inn}
\DeclareMathOperator{\Out}{Out}
\DeclareMathOperator{\Iso}{Iso}
\DeclareMathOperator{\Ob}{Ob}
\newcommand{\cop}[1]{{#1}^{op}}

% Triangle 
\newcommand{\tri}{\triangle}

% Partial derivative
\newcommand{\pa}{\partial}

% 4. Physics & Chemistry

% Quantum: h-bar
\newcommand{\hb}{\hbar}

% Partial operator
\newcommand{\pr}{\partial}



\endinput
\begin{document}
\def\htitle{Demo of My \LaTeX\ Style}
\def\hauthor{Hassium}
\hsetup
\htoc
\hmain
\section{Packages}
This style contains the following packages:
\begin{verbatim}
    \usepackage[T1]{fontenc}
    \usepackage[explicit]{titlesec}
    \usepackage[utf8]{inputenc}
    \usepackage{amsmath,amsthm,amssymb,amsfonts,mathrsfs,mathtools,nicematrix,chngcntr,
    centernot,ytableau,tikz-cd}
    \usepackage{imakeidx,textcomp,tocloft,environ,setspace,geometry,enumerate,
    enumitem,blindtext,multicol,xcolor,fancyhdr,calligra,graphicx,wrapfig,pgfplots,
    mdframed,tabularx,lipsum,comment,csquotes,verbatim,transparent,scalerel,halloweenmath}
    \usepackage[hidelinks]{hyperref}
    \usepackage{chemfig}
\end{verbatim}
How to insert it? 
\begin{verbatim}
    \documentclass{article} 
    %%% This is a scheme of a simple package. %%%
% version 1.0.6

% 1. Packages
\usepackage[T1]{fontenc}
\usepackage[hidelinks]{hyperref}
\usepackage[explicit]{titlesec}
\usepackage[utf8]{inputenc}
\usepackage{amsmath,amsthm,amssymb,amsfonts,mathrsfs,mathtools,nicematrix,chngcntr,centernot,ytableau,tikz-cd}
\usepackage{environ,setspace,geometry,enumerate,enumitem,blindtext,multicol,xcolor,calligra,graphicx,wrapfig,pgfplots,mdframed,tabularx,lipsum,comment,csquotes}
\usepackage{chemfig}

% 2. General Commands

% Enable useless warnings
% chktex-file 1 
% chktex-file 36 
% chktex-file 12 
% chktex-file 26 
% chktex-file 18

% Multicolumn setup
\setlength{\columnseprule}{1pt}
\def\columnseprulecolor{\color{black}}

% Geometry 
\geometry{letterpaper, margin=0.75in}
\setstretch{1.25}

% Newsection (shown in ToC, no counter)
\makeatletter
\newcommand\newsection[1]{%
  \section*{#1}%
  \addcontentsline{toc}{section}{#1}%
}
\makeatother

% Backslash
\newcommand{\bs}{\backslash}

% Hyperlink on ToC and section titles
\titleformat{\section}
{\normalfont\Large\bfseries}{\thesection}{1em}{\hyperlink{sec-\thesection}{#1}
\addtocontents{toc}{\protect\hypertarget{sec-\thesection}{}}}
\titleformat{name=\section,numberless}
{\normalfont\Large\bfseries}{}{0pt}{\hyperlink{toc}{#1}}

% Table of contents section only
\setcounter{tocdepth}{1}

% Changefont
\newcommand{\cfd}[1]{\fontfamily{pzc}\selectfont{#1}\fontfamily{cmr}\selectfont{}} 
\newcommand{\cfc}[1]{\calligra{#1}\fontfamily{cmr}\selectfont{}} 

% Enumerate
\setlist[enumerate]{topsep=0pt,itemsep=-1ex,partopsep=1ex,parsep=1ex}

% Remove proofs by empty space
\NewEnviron{reviewmode}{%
    \let\visibleproof\proof
    \let\endvisibleproof\endproof
    \RenewEnviron{proof}{\phantom{}\\\ \\ \\}{}
    \BODY
    \let\proof\visibleproof
    \let\endproof\endvisibleproof
}

% Darkmode (black background, white text)
\newcommand{\darkmode}{\pagecolor{black}\color{white}}

% Enumerate with tab
\newenvironment{level}
{\addtolength{\itemindent}{2em}}
{\addtolength{\itemindent}{-2em}}

% Circle
\newcommand*\circled[1]{\tikz[baseline=(char.base)]{\node[shape=circle,draw,inner sep=0.5pt](char){#1};}}

% Roman numberals
\newcommand{\rom}{\romannumeral}

% Pgfplot setup
\pgfplotsset{compat=1.18}

% 3. Math

% Theorem styles
\theoremstyle{definition}
\newtheorem{definition}{Definition}[section]
\newtheorem{theorem}{Theorem}[section]
\newtheorem*{proposition}{Proposition}
\newtheorem*{lemma}{Lemma}
\newtheorem*{corollary}{Corollary}
\newtheorem*{example}{Example}
\newtheorem*{remark}{Remark}
\newtheorem*{notation}{Notation}
\newtheorem{questioninner}{Exercise}
\newenvironment{exercise}[1][]{%
    \ifx\relax#1\relax\else\renewcommand{\thequestioninner}{#1}\fi 
    \questioninner
}{%
}
\makeatletter % title: exercise, with customize []
\newenvironment{x}[1]{%
    \par\medskip\noindent\textbf{#1\@addpunct{.}}\hskip\labelsep
}{%
    \par 
}
\makeatother % general environment for any title you want

% Equation counter
\counterwithin*{equation}{section}
\counterwithin*{equation}{subsection}

% Quiver (Authors: varkor (https://github.com/varkor), AndréC (https://tex.stackexchange.com/users/138900/andr%C3%A9c))
\usetikzlibrary{calc}
\usetikzlibrary{decorations.pathmorphing}
\tikzset{curve/.style={settings={#1},to path={(\tikztostart)
    .. controls ($(\tikztostart)!\pv{pos}!(\tikztotarget)!\pv{height}!270:(\tikztotarget)$)
    and ($(\tikztostart)!1-\pv{pos}!(\tikztotarget)!\pv{height}!270:(\tikztotarget)$)
    .. (\tikztotarget)\tikztonodes}},
    settings/.code={\tikzset{quiver/.cd,#1}
        \def\pv##1{\pgfkeysvalueof{/tikz/quiver/##1}}},
    quiver/.cd,pos/.initial=0.35,height/.initial=0}
\tikzset{tail reversed/.code={\pgfsetarrowsstart{tikzcd to}}}
\tikzset{2tail/.code={\pgfsetarrowsstart{Implies[reversed]}}}
\tikzset{2tail reversed/.code={\pgfsetarrowsstart{Implies}}}
\tikzset{no body/.style={/tikz/dash pattern=on 0 off 1mm}}

% Change font (math) 
\newcommand{\bb}[1]{\mathbb{#1}}
\newcommand{\ca}[1]{\mathcal{#1}}
\newcommand{\fr}[1]{\mathfrak{#1}}

% Natural, rational, real, complex numbers, integers
\newcommand{\N}{\mathbb{N}}
\newcommand{\C}{\mathbb{C}}
\newcommand{\R}{\mathbb{R}}
\newcommand{\Q}{\mathbb{Q}}
\newcommand{\Z}{\mathbb{Z}}

% Topological space
\newcommand{\T}{\mathcal{T}}

% Fancy symmetric group and alternating group
\newcommand{\Sg}{\mathfrak{S}}
\newcommand{\Ag}{\mathfrak{A}}

% Symmetric group
\DeclareMathOperator{\Sym}{Sym}

% Uniqueness
\newcommand{\uni}{\exists\ \text{!}\ }

% Greek and Hebrew letters
\newcommand{\al}{\alpha}
\newcommand{\be}{\beta}
\newcommand{\ga}{\gamma}
\newcommand{\ep}{\epsilon}
\newcommand{\de}{\delta}
\newcommand{\si}{\sigma}
\newcommand{\la}{\lambda}
\newcommand{\ka}{\kappa}
\newcommand{\vt}{\vartheta}
\newcommand{\vp}{\varphi}
\newcommand{\ve}{\varepsilon}

% Arrows, maps, morphisms
\newcommand{\ua}{\uparrow}
\newcommand{\da}{\downarrow}
\newcommand{\Ra}{\Rightarrow}
\newcommand{\La}{\Leftarrow}
\newcommand{\Ua}{\Uparrow}
\newcommand{\Da}{\Downarrow}
\newcommand{\nRa}{\nRightarrow}
\newcommand{\nLa}{\nLeftarrow}
\newcommand{\hra}{\hookrightarrow}
\newcommand{\hla}{\hookleftarrow}
\newcommand{\lt}{\leadsto}
\newcommand{\mt}{\mapsto}
\newcommand{\rat}{\rightarrowtail}
\newcommand{\lat}{\leftarrowtail}
\DeclareMathOperator{\id}{id}
\newcommand{\bij}{\xrightarrow{\sim}} % right arrow with \sim on it

% Sets, inclusions
\newcommand{\sub}{\subset}
\newcommand{\sube}{\subseteq}
\newcommand{\supe}{\supseteq}
\newcommand{\nsub}{\centernot\subset}
\newcommand{\nsup}{\centernot\supset}
\newcommand{\nsube}{\centernot\subseteq}
\newcommand{\nsupe}{\centernot\supseteq}
\newcommand{\es}{\varnothing}
\newcommand{\sm}{\setminus}
\newcommand{\ps}{\mathscr{P}}

% Otimes, oplus
\newcommand{\ot}{\otimes}
\newcommand{\op}{\oplus}

% Sign of permutation
\DeclareMathOperator{\sgn}{sgn}

% Normal Subgroup
\newcommand{\nsg}{\trianglelefteq}

% Defined as
\newcommand{\defa}{\coloneqq}

% Semidirect product
\newcommand{\sdp}{\rtimes}

% Inverse
\newcommand{\inv}[1]{{#1}^{-1}}

% Union, intersection, disjoint union
\newcommand{\Cap}{\bigcap}
\newcommand{\Cup}{\bigcup}
\newcommand{\DU}{\bigsqcup}

% Modulo
\renewcommand{\mod}{\ \text{mod}\ }

% Conjugacy classes
\DeclareMathOperator{\Cl}{Cl}

% Holomorph
\DeclareMathOperator{\Hol}{Hol}

% Composition of functions
\newcommand{\comp}{~\raisebox{1pt}{\tikz \draw[line width=0.6pt] circle(1.1pt);}~}

% Galois group
\DeclareMathOperator{\gal}{Gal}

% Cardinality
\newcommand{\card}[1]{\lvert{#1}\rvert}

% Norm
\newcommand{\norm}[1]{\lVert{#1}\rVert}

% Partial order
\newcommand{\po}{\preceq}

% Groups generated by
\newcommand{\cyc}[1]{\langle{#1}\rangle}

% Spectrum of a ring
\DeclareMathOperator{\Spec}{Spec}

% Sylow groups
\DeclareMathOperator{\Syl}{Syl}

% Category
\newcommand{\iso}{\approx}
\DeclareMathOperator{\Aut}{Aut}
\DeclareMathOperator{\End}{End}
\DeclareMathOperator{\Hom}{Hom}
\DeclareMathOperator{\Inn}{Inn}
\DeclareMathOperator{\Out}{Out}
\DeclareMathOperator{\Iso}{Iso}
\DeclareMathOperator{\Ob}{Ob}
\newcommand{\cop}[1]{{#1}^{op}}

% Triangle 
\newcommand{\tri}{\triangle}

% Partial derivative
\newcommand{\pa}{\partial}

% 4. Physics & Chemistry

% Quantum: h-bar
\newcommand{\hb}{\hbar}

% Partial operator
\newcommand{\pr}{\partial}



\endinput % Download and input it using its path 
\end{verbatim}
\section{Title Page Setup}
After inserting the package, you should define the title and author name. Here is an example, which is the code of this demo:
\begin{verbatim}
    \documentclass{article}
    %%% This is a scheme of a simple package. %%%
% version 1.0.6

% 1. Packages
\usepackage[T1]{fontenc}
\usepackage[hidelinks]{hyperref}
\usepackage[explicit]{titlesec}
\usepackage[utf8]{inputenc}
\usepackage{amsmath,amsthm,amssymb,amsfonts,mathrsfs,mathtools,nicematrix,chngcntr,centernot,ytableau,tikz-cd}
\usepackage{environ,setspace,geometry,enumerate,enumitem,blindtext,multicol,xcolor,calligra,graphicx,wrapfig,pgfplots,mdframed,tabularx,lipsum,comment,csquotes}
\usepackage{chemfig}

% 2. General Commands

% Enable useless warnings
% chktex-file 1 
% chktex-file 36 
% chktex-file 12 
% chktex-file 26 
% chktex-file 18

% Multicolumn setup
\setlength{\columnseprule}{1pt}
\def\columnseprulecolor{\color{black}}

% Geometry 
\geometry{letterpaper, margin=0.75in}
\setstretch{1.25}

% Newsection (shown in ToC, no counter)
\makeatletter
\newcommand\newsection[1]{%
  \section*{#1}%
  \addcontentsline{toc}{section}{#1}%
}
\makeatother

% Backslash
\newcommand{\bs}{\backslash}

% Hyperlink on ToC and section titles
\titleformat{\section}
{\normalfont\Large\bfseries}{\thesection}{1em}{\hyperlink{sec-\thesection}{#1}
\addtocontents{toc}{\protect\hypertarget{sec-\thesection}{}}}
\titleformat{name=\section,numberless}
{\normalfont\Large\bfseries}{}{0pt}{\hyperlink{toc}{#1}}

% Table of contents section only
\setcounter{tocdepth}{1}

% Changefont
\newcommand{\cfd}[1]{\fontfamily{pzc}\selectfont{#1}\fontfamily{cmr}\selectfont{}} 
\newcommand{\cfc}[1]{\calligra{#1}\fontfamily{cmr}\selectfont{}} 

% Enumerate
\setlist[enumerate]{topsep=0pt,itemsep=-1ex,partopsep=1ex,parsep=1ex}

% Remove proofs by empty space
\NewEnviron{reviewmode}{%
    \let\visibleproof\proof
    \let\endvisibleproof\endproof
    \RenewEnviron{proof}{\phantom{}\\\ \\ \\}{}
    \BODY
    \let\proof\visibleproof
    \let\endproof\endvisibleproof
}

% Darkmode (black background, white text)
\newcommand{\darkmode}{\pagecolor{black}\color{white}}

% Enumerate with tab
\newenvironment{level}
{\addtolength{\itemindent}{2em}}
{\addtolength{\itemindent}{-2em}}

% Circle
\newcommand*\circled[1]{\tikz[baseline=(char.base)]{\node[shape=circle,draw,inner sep=0.5pt](char){#1};}}

% Roman numberals
\newcommand{\rom}{\romannumeral}

% Pgfplot setup
\pgfplotsset{compat=1.18}

% 3. Math

% Theorem styles
\theoremstyle{definition}
\newtheorem{definition}{Definition}[section]
\newtheorem{theorem}{Theorem}[section]
\newtheorem*{proposition}{Proposition}
\newtheorem*{lemma}{Lemma}
\newtheorem*{corollary}{Corollary}
\newtheorem*{example}{Example}
\newtheorem*{remark}{Remark}
\newtheorem*{notation}{Notation}
\newtheorem{questioninner}{Exercise}
\newenvironment{exercise}[1][]{%
    \ifx\relax#1\relax\else\renewcommand{\thequestioninner}{#1}\fi 
    \questioninner
}{%
}
\makeatletter % title: exercise, with customize []
\newenvironment{x}[1]{%
    \par\medskip\noindent\textbf{#1\@addpunct{.}}\hskip\labelsep
}{%
    \par 
}
\makeatother % general environment for any title you want

% Equation counter
\counterwithin*{equation}{section}
\counterwithin*{equation}{subsection}

% Quiver (Authors: varkor (https://github.com/varkor), AndréC (https://tex.stackexchange.com/users/138900/andr%C3%A9c))
\usetikzlibrary{calc}
\usetikzlibrary{decorations.pathmorphing}
\tikzset{curve/.style={settings={#1},to path={(\tikztostart)
    .. controls ($(\tikztostart)!\pv{pos}!(\tikztotarget)!\pv{height}!270:(\tikztotarget)$)
    and ($(\tikztostart)!1-\pv{pos}!(\tikztotarget)!\pv{height}!270:(\tikztotarget)$)
    .. (\tikztotarget)\tikztonodes}},
    settings/.code={\tikzset{quiver/.cd,#1}
        \def\pv##1{\pgfkeysvalueof{/tikz/quiver/##1}}},
    quiver/.cd,pos/.initial=0.35,height/.initial=0}
\tikzset{tail reversed/.code={\pgfsetarrowsstart{tikzcd to}}}
\tikzset{2tail/.code={\pgfsetarrowsstart{Implies[reversed]}}}
\tikzset{2tail reversed/.code={\pgfsetarrowsstart{Implies}}}
\tikzset{no body/.style={/tikz/dash pattern=on 0 off 1mm}}

% Change font (math) 
\newcommand{\bb}[1]{\mathbb{#1}}
\newcommand{\ca}[1]{\mathcal{#1}}
\newcommand{\fr}[1]{\mathfrak{#1}}

% Natural, rational, real, complex numbers, integers
\newcommand{\N}{\mathbb{N}}
\newcommand{\C}{\mathbb{C}}
\newcommand{\R}{\mathbb{R}}
\newcommand{\Q}{\mathbb{Q}}
\newcommand{\Z}{\mathbb{Z}}

% Topological space
\newcommand{\T}{\mathcal{T}}

% Fancy symmetric group and alternating group
\newcommand{\Sg}{\mathfrak{S}}
\newcommand{\Ag}{\mathfrak{A}}

% Symmetric group
\DeclareMathOperator{\Sym}{Sym}

% Uniqueness
\newcommand{\uni}{\exists\ \text{!}\ }

% Greek and Hebrew letters
\newcommand{\al}{\alpha}
\newcommand{\be}{\beta}
\newcommand{\ga}{\gamma}
\newcommand{\ep}{\epsilon}
\newcommand{\de}{\delta}
\newcommand{\si}{\sigma}
\newcommand{\la}{\lambda}
\newcommand{\ka}{\kappa}
\newcommand{\vt}{\vartheta}
\newcommand{\vp}{\varphi}
\newcommand{\ve}{\varepsilon}

% Arrows, maps, morphisms
\newcommand{\ua}{\uparrow}
\newcommand{\da}{\downarrow}
\newcommand{\Ra}{\Rightarrow}
\newcommand{\La}{\Leftarrow}
\newcommand{\Ua}{\Uparrow}
\newcommand{\Da}{\Downarrow}
\newcommand{\nRa}{\nRightarrow}
\newcommand{\nLa}{\nLeftarrow}
\newcommand{\hra}{\hookrightarrow}
\newcommand{\hla}{\hookleftarrow}
\newcommand{\lt}{\leadsto}
\newcommand{\mt}{\mapsto}
\newcommand{\rat}{\rightarrowtail}
\newcommand{\lat}{\leftarrowtail}
\DeclareMathOperator{\id}{id}
\newcommand{\bij}{\xrightarrow{\sim}} % right arrow with \sim on it

% Sets, inclusions
\newcommand{\sub}{\subset}
\newcommand{\sube}{\subseteq}
\newcommand{\supe}{\supseteq}
\newcommand{\nsub}{\centernot\subset}
\newcommand{\nsup}{\centernot\supset}
\newcommand{\nsube}{\centernot\subseteq}
\newcommand{\nsupe}{\centernot\supseteq}
\newcommand{\es}{\varnothing}
\newcommand{\sm}{\setminus}
\newcommand{\ps}{\mathscr{P}}

% Otimes, oplus
\newcommand{\ot}{\otimes}
\newcommand{\op}{\oplus}

% Sign of permutation
\DeclareMathOperator{\sgn}{sgn}

% Normal Subgroup
\newcommand{\nsg}{\trianglelefteq}

% Defined as
\newcommand{\defa}{\coloneqq}

% Semidirect product
\newcommand{\sdp}{\rtimes}

% Inverse
\newcommand{\inv}[1]{{#1}^{-1}}

% Union, intersection, disjoint union
\newcommand{\Cap}{\bigcap}
\newcommand{\Cup}{\bigcup}
\newcommand{\DU}{\bigsqcup}

% Modulo
\renewcommand{\mod}{\ \text{mod}\ }

% Conjugacy classes
\DeclareMathOperator{\Cl}{Cl}

% Holomorph
\DeclareMathOperator{\Hol}{Hol}

% Composition of functions
\newcommand{\comp}{~\raisebox{1pt}{\tikz \draw[line width=0.6pt] circle(1.1pt);}~}

% Galois group
\DeclareMathOperator{\gal}{Gal}

% Cardinality
\newcommand{\card}[1]{\lvert{#1}\rvert}

% Norm
\newcommand{\norm}[1]{\lVert{#1}\rVert}

% Partial order
\newcommand{\po}{\preceq}

% Groups generated by
\newcommand{\cyc}[1]{\langle{#1}\rangle}

% Spectrum of a ring
\DeclareMathOperator{\Spec}{Spec}

% Sylow groups
\DeclareMathOperator{\Syl}{Syl}

% Category
\newcommand{\iso}{\approx}
\DeclareMathOperator{\Aut}{Aut}
\DeclareMathOperator{\End}{End}
\DeclareMathOperator{\Hom}{Hom}
\DeclareMathOperator{\Inn}{Inn}
\DeclareMathOperator{\Out}{Out}
\DeclareMathOperator{\Iso}{Iso}
\DeclareMathOperator{\Ob}{Ob}
\newcommand{\cop}[1]{{#1}^{op}}

% Triangle 
\newcommand{\tri}{\triangle}

% Partial derivative
\newcommand{\pa}{\partial}

% 4. Physics & Chemistry

% Quantum: h-bar
\newcommand{\hb}{\hbar}

% Partial operator
\newcommand{\pr}{\partial}



\endinput
    \begin{document}
        \def\htitle{Demo of Hassium Style}
        \def\hauthor{Hassium}
        \hsetup
        \htoc
        \hmain
    \end{document}
\end{verbatim}
\section{Page Geometry}
There are some commands that adjust the geometry of the document:
\begin{verbatim}
    \geometry{letterpaper, top=54pt,bottom=46.8pt,marginparsep=5.67pt,marginparwidth=56.69pt,
    voffset=0pt,hoffset=0pt,left=54pt,right=54pt,headheight=24pt,headsep=10pt}
    \setstretch{1.25} % spacing
\end{verbatim}
\section{More on Table of Contents}
You can add descriptions to each section and the description will appear in the table of contents, directly below the section name: 
\begin{verbatim}
    \section{This is a Sample Section} 
    \descr{This is a description to the section} 
\end{verbatim}
\noindent
The table of contents only shows the section names, but no subsections and numberless sections. If you want a numberless section in the table of contents, use the ``newsection'' command:
\begin{verbatim}
    \newsection{This is a numberless section} 
\end{verbatim}
\noindent
Note that the section names in the table of contents are hyperlinks; click on any section name to navigate directly to that section. You can do the converse to navigate to the first page as well.
\section{Index Page}
This style has a customized index page. Check the code:
\begin{verbatim}
    This is a \hdef{defintiion}. This is another \hdef{vocabulary}.
    \hindex
\end{verbatim}
The command ``hdef'' mark the word and print it. The command ``hindex'' is a customized index page that print words in three columns. Each page number in the index page contains a hyperlink to that page.
\section{Darkmode}
Darkmode command changes the background color to black and the text to white.
\begin{verbatim}
    \begin{document}
        \darkmode 
    \end{document}
\end{verbatim}
\section{Other Environments and Commands}
The line-spacing in ``enumerate'' environment is changed:
\begin{verbatim}
    \setlist[enumerate]{topsep=0pt,itemsep=-1ex,partopsep=1ex,parsep=1ex}
\end{verbatim}
\noindent
The ``level'' environment is used in ``enumerate'' environment, consider the following code:
\begin{verbatim}
    \begin{enumerate}
        \item This is the first line.
        \begin{level}
            \item This is the second line.
            \begin{level}
                \item This is the third line.
            \end{level}
            \item This is another line.
        \end{level}
    \end{enumerate}
\end{verbatim}
\noindent
This code gives:
\begin{enumerate}
    \item This is the first line.
    \begin{level}
        \item This is the second line.
        \begin{level}
            \item This is the third line.
        \end{level}
        \item This is another line.
    \end{level}
\end{enumerate}
\noindent
The command ``circled'' draws a small circle and you can add something inside the circle:
\begin{verbatim}
    \circled{1}
\end{verbatim}
\noindent
The output is $\circled{1}$. You can write any Romam numerals by:
\begin{verbatim}
    \rom108 
\end{verbatim}
\noindent
There are two simple commands for hand-written fonts:
\begin{verbatim}
    \cfd{font 1}
    \cfc{font 2}
\end{verbatim}
The outputs are \cfd{font 1} and \cfc{font 2}.
\section{Quiver}
Quiver is done by \href{https://github.com/varkor}{varkor} and \href{https://tex.stackexchange.com/users/138900/andr%C3%A9c}{AndréC}, check their github for more information. I include quiver to draw curve arrows in a commutative diagram. To draw a diagram with quiver, check this \href{https://q.uiver.app/}{website}. An example is given below:
\begin{verbatim}
    % chktex-file 15 % the three lines enables useless warnings
    % chktex-file 17
    % chktex-file 18
    \begin{center}
        \begin{tikzcd}
            Hello &&&& World \\
            \\
            \\
            && Hassium
            \arrow["\shortmid"{marking}, curve={height=-6pt}, tail reversed, from=1-1, to=1-5]
            \arrow[curve={height=6pt}, squiggly, from=1-1, to=4-3]
            \arrow[curve={height=-6pt}, dashed, hook', from=1-5, to=4-3]
        \end{tikzcd}
    \end{center}
\end{verbatim}
The diagram looks like:
\begin{center}
    $\begin{tikzcd}
        Hello &&&& World \\
        \\
        \\
        && Hassium
        \arrow["\shortmid"{marking}, curve={height=-6pt}, tail reversed, from=1-1, to=1-5]
        \arrow[curve={height=6pt}, squiggly, from=1-1, to=4-3]
        \arrow[curve={height=-6pt}, dashed, hook', from=1-5, to=4-3]
    \end{tikzcd}$
\end{center}
\section{Theorem Styles}
Several theorem styles are offered:
\begin{verbatim}
    \theoremstyle{definition}
    \newtheorem{definition}{Definition}[section]
    \newtheorem{theorem}{Theorem}[section]
    \newtheorem*{proposition}{Proposition}
    \newtheorem*{lemma}{Lemma}
    \newtheorem*{corollary}{Corollary}
    \newtheorem*{example}{Example}
    \newtheorem*{remark}{Remark}
    \newtheorem*{notation}{Notation}
\end{verbatim}
\noindent
There is a ``hdefinition'' environment, which works exactly the same as ``definition'' if you write:
\begin{verbatim}
    \begin{hdefinition}
        This is a definition of Hassium.
    \end{hdefinition}
\end{verbatim}
\noindent
If you include a name variable, it gives an index to the name.
\begin{verbatim}
    \begin{hdefinition}[Hassium]
        This is a definition of Hassium
    \end{hdefinition}
    \hindex % This will print Hassium
\end{verbatim} 
\noindent
The environment name can be customized by using:
\begin{verbatim}
    \customtheorem{This is a custom theorem}
    \begin{This is a custom theorem}
        The proof is trivial.
    \end{This is a custom theorem}
\end{verbatim}
\noindent
The output environment is:
\customtheorem{This is a custom theorem}
\begin{This is a custom theorem}
    The proof is trivial.
\end{This is a custom theorem}
\noindent
You can put any number or label in ``exercise'' environment:
\begin{verbatim}
    \begin{exercise}[8.6]
        The proof is trivial.
    \end{exercise}
\end{verbatim}
\noindent
The environment looks like:
\begin{exercise}[8.6]
    The proof is trivial.
\end{exercise}
\section{Invisible Proofs}
The environment ``reviewmode'' is originally done by my friend \href{https://github.com/ETwilight}{ETwilight}. It replaces your ``proof'' environment by three empty lines:
\begin{verbatim}
    \begin{reviewmode}
        \begin{proof}
            The proof is trivial.
        \end{proof}
    \end{reviewmode}
\end{verbatim}
\section{Simple Commands in Math Mode}
I will give a table of all commands in math mode.
\setlength{\columnseprule}{1pt}
\def\columnseprulecolor{\color{black}}
\begin{multicols}{2}
    \noindent$\bs$bs\hfill$\bs$ \\
    $\bs$N\hfill$\N$ \\
    $\bs$Z\hfill$\Z$ \\
    $\bs$Q\hfill$\Q$ \\
    $\bs$R\hfill$\R$ \\
    $\bs$C\hfill$\C$ \\
    $\bs$bb$\{$H$\}$\hfill$\bb{H}$ \\
    $\bs$ca$\{$H$\}$\hfill$\ca{H}$ \\
    $\bs$fr$\{$H$\}$\hfill$\fr{H}$ \\
    $\bs$T\hfill$\T$ \\
    $\bs$Pn$\{$n$\}$\hfill$\Pn{n}$ \\
    $\bs$CP$\{$n$\}$\hfill$\CP{n}$ \\
    $\bs$RP$\{$n$\}$\hfill$\RP{n}$ \\
    $\bs$Sym\hfill$\Sym$ \\
    $\bs$GL\hfill$\GL$ \\
    $\bs$SL\hfill$\SL$ \\
    $\bs$Mod\hfill$\Mod$ \\
    $\bs$Sg\hfill$\Sg$ \\
    $\bs$Ag\hfill$\Ag$ \\
    $\bs$Cay\hfill$\Cay$ \\
    $\bs$uni\hfill$\uni$ \\
    $\bs$al\hfill$\al$ \\
    $\bs$be\hfill$\be$ \\
    $\bs$ga\hfill$\ga$ \\
    $\bs$de\hfill$\de$ \\
    $\bs$ep\hfill$\ep$ \\
    $\bs$si\hfill$\si$ \\
    $\bs$la\hfill$\la$ \\
    $\bs$ka\hfill$\ka$ \\
    $\bs$om\hfill$\om$ \\
    $\bs$Ga\hfill$\Ga$ \\
    $\bs$De\hfill$\De$ \\
    $\bs$Si\hfill$\Si$ \\
    $\bs$LA\hfill$\LA$ \\
    $\bs$Om\hfill$\Om$ \\
    $\bs$vp\hfill$\vp$ \\
    $\bs$vt\hfill$\vt$ \\
    $\bs$ve\hfill$\ve$ \\
    $\bs$ua\hfill$\ua$ \\
    $\bs$da\hfill$\da$ \\
    $\bs$Ra\hfill$\Ra$ \\
    $\bs$La\hfill$\La$ \\
    $\bs$Ua\hfill$\Ua$ \\
    $\bs$Da\hfill$\Da$ \\
    $\bs$nRa\hfill$\nRa$ \\
    $\bs$nLa\hfill$\nLa$ \\
    $\bs$hra\hfill$\hra$ \\
    $\bs$hla\hfill$\hla$ \\
    $\bs$lt\hfill$\lt$ \\
    $\bs$mt\hfill$\mt$ \\
    $\bs$rat\hfill$\rat$ \\
    $\bs$lat\hfill$\lat$ \\
    $\bs$thra\hfill$\thra$ \\
    $\bs$thla\hfill$\thla$ \\
    $\bs$bij\hfill$\bij$ \\
    $\bs$wb$\{$A$\}$\hfill$\wb{A}$ \\
    $\bs$id\hfill$\id$ \\
    $\bs$sub\hfill$\sub$ \\
    $\bs$sube\hfill$\sube$ \\
    $\bs$supe\hfill$\supe$ \\
    $\bs$nsub\hfill$\nsub$ \\
    $\bs$nsup\hfill$\nsup$ \\
    $\bs$nsube\hfill$\nsube$ \\
    $\bs$nsupe\hfill$\nsupe$ \\
    $\bs$subn\hfill$\subn$ \\
    $\bs$supn\hfill$\supn$ \\
    $\bs$es\hfill$\es$ \\
    $\bs$sm\hfill$\sm$ \\
    $\bs$ps\hfill$\ps$ \\
    $\bs$Un\hfill$\Un$ \\
    $\bs$In\hfill$\In$ \\
    $\bs$Du\hfill$\Du$ \\
    $\bs$cp\hfill$\cp$ \\
    $\bs$Cp\hfill$\Cp$ \\
    $\bs$ot\hfill$\ot$ \\
    $\bs$op\hfill$\op$ \\
    $\bs$acts\hfill$\acts$ \\
    $\bs$Span\hfill$\Span$ \\
    $\bs$sgn\hfill$\sgn$ \\
    $\bs$nsg\hfill$\nsg$ \\
    $\bs$defa\hfill$\defa$ \\
    $\bs$sdp\hfill$\sdp$ \\
    $\bs$inv$\{$f$\}$\hfill$\inv{f}$ \\
    x$\bs$mod y\hfill$x\mod y$ \\
    $\bs$Cl\hfill$\Cl$ \\
    $\bs$Hol\hfill$\Hol$ \\
    $\bs$comp\hfill$\comp$ \\
    $\bs$Gal\hfill$\Gal$ \\
    $\bs$card$\{S\}$\hfill$\card{S}$ \\
    $\bs$im\hfill$\im$ \\
    $\bs$norm$\{$M$\}$\hfill$\norm{M}$ \\
    $\bs$po\hfill$\po$ \\
    $\bs$poe\hfill$\poe$ \\
    $\bs$cyc$\{$g$\}$\hfill$\cyc{g}$ \\
    $\bs$Spec\hfill$\Spec$ \\
    $\bs$Syl\hfill$\Syl$ \\
    $\bs$iso\hfill$\iso$ \\
    $\bs$niso\hfill$\niso$ \\
    $\bs$Mor\hfill$\Mor$ \\
    $\bs$Aut\hfill$\Aut$ \\
    $\bs$End\hfill$\End$ \\
    $\bs$Hom\hfill$\Hom$ \\
    $\bs$Inn\hfill$\Inn$ \\
    $\bs$Out\hfill$\Out$ \\
    $\bs$Iso\hfill$\Iso$ \\
    $\bs$Ob\hfill$\Ob$ \\
    $\bs$cop$\{$C$\}$\hfill$\cop{C}$ \\
    $\bs$tri\hfill$\tri$ \\
    $\bs$pa\hfill$\pa$ \\
    $\bs$Ann\hfill$\Ann$ \\
    $\bs$dom\hfill$\dom$ \\
    $\bs$ran\hfill$\ran$ \\
    $\bs$cod\hfill$\cod$ \\
    $\bs$A$\{$n$\}$\hfill$\A{n}$ \\
    $\bs$sq\hfill$\sq$ \\
    $\bs$CAT\hfill$\CAT$ \\
    $\bs$fl$\{$A$\}$\hfill$\fl{A}$ \\
    $\bs$'\hfill$\'$
\end{multicols}
\section{Acknowledgement}
Special thanks to \cfd{FSG}; his advice on style has been invaluable.
\hindex
\end{document}