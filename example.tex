\documentclass[10pt]{article}
%%% This is a scheme of a simple package. %%%
% version 1.0.6

% 1. Packages
\usepackage[T1]{fontenc}
\usepackage[hidelinks]{hyperref}
\usepackage[explicit]{titlesec}
\usepackage[utf8]{inputenc}
\usepackage{amsmath,amsthm,amssymb,amsfonts,mathrsfs,mathtools,nicematrix,chngcntr,centernot,ytableau,tikz-cd}
\usepackage{environ,setspace,geometry,enumerate,enumitem,blindtext,multicol,xcolor,calligra,graphicx,wrapfig,pgfplots,mdframed,tabularx,lipsum,comment,csquotes}
\usepackage{chemfig}

% 2. General Commands

% Enable useless warnings
% chktex-file 1 
% chktex-file 36 
% chktex-file 12 
% chktex-file 26 
% chktex-file 18

% Multicolumn setup
\setlength{\columnseprule}{1pt}
\def\columnseprulecolor{\color{black}}

% Geometry 
\geometry{letterpaper, margin=0.75in}
\setstretch{1.25}

% Newsection (shown in ToC, no counter)
\makeatletter
\newcommand\newsection[1]{%
  \section*{#1}%
  \addcontentsline{toc}{section}{#1}%
}
\makeatother

% Backslash
\newcommand{\bs}{\backslash}

% Hyperlink on ToC and section titles
\titleformat{\section}
{\normalfont\Large\bfseries}{\thesection}{1em}{\hyperlink{sec-\thesection}{#1}
\addtocontents{toc}{\protect\hypertarget{sec-\thesection}{}}}
\titleformat{name=\section,numberless}
{\normalfont\Large\bfseries}{}{0pt}{\hyperlink{toc}{#1}}

% Table of contents section only
\setcounter{tocdepth}{1}

% Changefont
\newcommand{\cfd}[1]{\fontfamily{pzc}\selectfont{#1}\fontfamily{cmr}\selectfont{}} 
\newcommand{\cfc}[1]{\calligra{#1}\fontfamily{cmr}\selectfont{}} 

% Enumerate
\setlist[enumerate]{topsep=0pt,itemsep=-1ex,partopsep=1ex,parsep=1ex}

% Remove proofs by empty space
\NewEnviron{reviewmode}{%
    \let\visibleproof\proof
    \let\endvisibleproof\endproof
    \RenewEnviron{proof}{\phantom{}\\\ \\ \\}{}
    \BODY
    \let\proof\visibleproof
    \let\endproof\endvisibleproof
}

% Darkmode (black background, white text)
\newcommand{\darkmode}{\pagecolor{black}\color{white}}

% Enumerate with tab
\newenvironment{level}
{\addtolength{\itemindent}{2em}}
{\addtolength{\itemindent}{-2em}}

% Circle
\newcommand*\circled[1]{\tikz[baseline=(char.base)]{\node[shape=circle,draw,inner sep=0.5pt](char){#1};}}

% Roman numberals
\newcommand{\rom}{\romannumeral}

% Pgfplot setup
\pgfplotsset{compat=1.18}

% 3. Math

% Theorem styles
\theoremstyle{definition}
\newtheorem{definition}{Definition}[section]
\newtheorem{theorem}{Theorem}[section]
\newtheorem*{proposition}{Proposition}
\newtheorem*{lemma}{Lemma}
\newtheorem*{corollary}{Corollary}
\newtheorem*{example}{Example}
\newtheorem*{remark}{Remark}
\newtheorem*{notation}{Notation}
\newtheorem{questioninner}{Exercise}
\newenvironment{exercise}[1][]{%
    \ifx\relax#1\relax\else\renewcommand{\thequestioninner}{#1}\fi 
    \questioninner
}{%
}
\makeatletter % title: exercise, with customize []
\newenvironment{x}[1]{%
    \par\medskip\noindent\textbf{#1\@addpunct{.}}\hskip\labelsep
}{%
    \par 
}
\makeatother % general environment for any title you want

% Equation counter
\counterwithin*{equation}{section}
\counterwithin*{equation}{subsection}

% Quiver (Authors: varkor (https://github.com/varkor), AndréC (https://tex.stackexchange.com/users/138900/andr%C3%A9c))
\usetikzlibrary{calc}
\usetikzlibrary{decorations.pathmorphing}
\tikzset{curve/.style={settings={#1},to path={(\tikztostart)
    .. controls ($(\tikztostart)!\pv{pos}!(\tikztotarget)!\pv{height}!270:(\tikztotarget)$)
    and ($(\tikztostart)!1-\pv{pos}!(\tikztotarget)!\pv{height}!270:(\tikztotarget)$)
    .. (\tikztotarget)\tikztonodes}},
    settings/.code={\tikzset{quiver/.cd,#1}
        \def\pv##1{\pgfkeysvalueof{/tikz/quiver/##1}}},
    quiver/.cd,pos/.initial=0.35,height/.initial=0}
\tikzset{tail reversed/.code={\pgfsetarrowsstart{tikzcd to}}}
\tikzset{2tail/.code={\pgfsetarrowsstart{Implies[reversed]}}}
\tikzset{2tail reversed/.code={\pgfsetarrowsstart{Implies}}}
\tikzset{no body/.style={/tikz/dash pattern=on 0 off 1mm}}

% Change font (math) 
\newcommand{\bb}[1]{\mathbb{#1}}
\newcommand{\ca}[1]{\mathcal{#1}}
\newcommand{\fr}[1]{\mathfrak{#1}}

% Natural, rational, real, complex numbers, integers
\newcommand{\N}{\mathbb{N}}
\newcommand{\C}{\mathbb{C}}
\newcommand{\R}{\mathbb{R}}
\newcommand{\Q}{\mathbb{Q}}
\newcommand{\Z}{\mathbb{Z}}

% Topological space
\newcommand{\T}{\mathcal{T}}

% Fancy symmetric group and alternating group
\newcommand{\Sg}{\mathfrak{S}}
\newcommand{\Ag}{\mathfrak{A}}

% Symmetric group
\DeclareMathOperator{\Sym}{Sym}

% Uniqueness
\newcommand{\uni}{\exists\ \text{!}\ }

% Greek and Hebrew letters
\newcommand{\al}{\alpha}
\newcommand{\be}{\beta}
\newcommand{\ga}{\gamma}
\newcommand{\ep}{\epsilon}
\newcommand{\de}{\delta}
\newcommand{\si}{\sigma}
\newcommand{\la}{\lambda}
\newcommand{\ka}{\kappa}
\newcommand{\vt}{\vartheta}
\newcommand{\vp}{\varphi}
\newcommand{\ve}{\varepsilon}

% Arrows, maps, morphisms
\newcommand{\ua}{\uparrow}
\newcommand{\da}{\downarrow}
\newcommand{\Ra}{\Rightarrow}
\newcommand{\La}{\Leftarrow}
\newcommand{\Ua}{\Uparrow}
\newcommand{\Da}{\Downarrow}
\newcommand{\nRa}{\nRightarrow}
\newcommand{\nLa}{\nLeftarrow}
\newcommand{\hra}{\hookrightarrow}
\newcommand{\hla}{\hookleftarrow}
\newcommand{\lt}{\leadsto}
\newcommand{\mt}{\mapsto}
\newcommand{\rat}{\rightarrowtail}
\newcommand{\lat}{\leftarrowtail}
\DeclareMathOperator{\id}{id}
\newcommand{\bij}{\xrightarrow{\sim}} % right arrow with \sim on it

% Sets, inclusions
\newcommand{\sub}{\subset}
\newcommand{\sube}{\subseteq}
\newcommand{\supe}{\supseteq}
\newcommand{\nsub}{\centernot\subset}
\newcommand{\nsup}{\centernot\supset}
\newcommand{\nsube}{\centernot\subseteq}
\newcommand{\nsupe}{\centernot\supseteq}
\newcommand{\es}{\varnothing}
\newcommand{\sm}{\setminus}
\newcommand{\ps}{\mathscr{P}}

% Otimes, oplus
\newcommand{\ot}{\otimes}
\newcommand{\op}{\oplus}

% Sign of permutation
\DeclareMathOperator{\sgn}{sgn}

% Normal Subgroup
\newcommand{\nsg}{\trianglelefteq}

% Defined as
\newcommand{\defa}{\coloneqq}

% Semidirect product
\newcommand{\sdp}{\rtimes}

% Inverse
\newcommand{\inv}[1]{{#1}^{-1}}

% Union, intersection, disjoint union
\newcommand{\Cap}{\bigcap}
\newcommand{\Cup}{\bigcup}
\newcommand{\DU}{\bigsqcup}

% Modulo
\renewcommand{\mod}{\ \text{mod}\ }

% Conjugacy classes
\DeclareMathOperator{\Cl}{Cl}

% Holomorph
\DeclareMathOperator{\Hol}{Hol}

% Composition of functions
\newcommand{\comp}{~\raisebox{1pt}{\tikz \draw[line width=0.6pt] circle(1.1pt);}~}

% Galois group
\DeclareMathOperator{\gal}{Gal}

% Cardinality
\newcommand{\card}[1]{\lvert{#1}\rvert}

% Norm
\newcommand{\norm}[1]{\lVert{#1}\rVert}

% Partial order
\newcommand{\po}{\preceq}

% Groups generated by
\newcommand{\cyc}[1]{\langle{#1}\rangle}

% Spectrum of a ring
\DeclareMathOperator{\Spec}{Spec}

% Sylow groups
\DeclareMathOperator{\Syl}{Syl}

% Category
\newcommand{\iso}{\approx}
\DeclareMathOperator{\Aut}{Aut}
\DeclareMathOperator{\End}{End}
\DeclareMathOperator{\Hom}{Hom}
\DeclareMathOperator{\Inn}{Inn}
\DeclareMathOperator{\Out}{Out}
\DeclareMathOperator{\Iso}{Iso}
\DeclareMathOperator{\Ob}{Ob}
\newcommand{\cop}[1]{{#1}^{op}}

% Triangle 
\newcommand{\tri}{\triangle}

% Partial derivative
\newcommand{\pa}{\partial}

% 4. Physics & Chemistry

% Quantum: h-bar
\newcommand{\hb}{\hbar}

% Partial operator
\newcommand{\pr}{\partial}



\endinput
\begin{document}
\def\htitle{Demo of Hassium Style}
\def\hauthor{Hassium}
\hsetup\
\htoc\
\hmain\
\section{Packages and General Setup}
This style contains the following packages:
\begin{verbatim}
    \usepackage[T1]{fontenc}
    \usepackage[hidelinks]{hyperref}
    \usepackage[explicit]{titlesec}
    \usepackage[utf8]{inputenc}
    \usepackage{amsmath,amsthm,amssymb,amsfonts,mathrsfs,mathtools,nicematrix,chngcntr,
    centernot,ytableau,tikz-cd}
    \usepackage{textcomp,tocloft,environ,setspace,geometry,enumerate,enumitem,blindtext,
    multicol,xcolor,fancyhdr,calligra,graphicx,wrapfig,pgfplots,mdframed,tabularx,lipsum,
    comment,csquotes}
    \usepackage{chemfig}
\end{verbatim}
How to insert it? 
\begin{verbatim}
    \documentclass{article} % This style only has commands on \section
    %%% This is a scheme of a simple package. %%%
% version 1.0.6

% 1. Packages
\usepackage[T1]{fontenc}
\usepackage[hidelinks]{hyperref}
\usepackage[explicit]{titlesec}
\usepackage[utf8]{inputenc}
\usepackage{amsmath,amsthm,amssymb,amsfonts,mathrsfs,mathtools,nicematrix,chngcntr,centernot,ytableau,tikz-cd}
\usepackage{environ,setspace,geometry,enumerate,enumitem,blindtext,multicol,xcolor,calligra,graphicx,wrapfig,pgfplots,mdframed,tabularx,lipsum,comment,csquotes}
\usepackage{chemfig}

% 2. General Commands

% Enable useless warnings
% chktex-file 1 
% chktex-file 36 
% chktex-file 12 
% chktex-file 26 
% chktex-file 18

% Multicolumn setup
\setlength{\columnseprule}{1pt}
\def\columnseprulecolor{\color{black}}

% Geometry 
\geometry{letterpaper, margin=0.75in}
\setstretch{1.25}

% Newsection (shown in ToC, no counter)
\makeatletter
\newcommand\newsection[1]{%
  \section*{#1}%
  \addcontentsline{toc}{section}{#1}%
}
\makeatother

% Backslash
\newcommand{\bs}{\backslash}

% Hyperlink on ToC and section titles
\titleformat{\section}
{\normalfont\Large\bfseries}{\thesection}{1em}{\hyperlink{sec-\thesection}{#1}
\addtocontents{toc}{\protect\hypertarget{sec-\thesection}{}}}
\titleformat{name=\section,numberless}
{\normalfont\Large\bfseries}{}{0pt}{\hyperlink{toc}{#1}}

% Table of contents section only
\setcounter{tocdepth}{1}

% Changefont
\newcommand{\cfd}[1]{\fontfamily{pzc}\selectfont{#1}\fontfamily{cmr}\selectfont{}} 
\newcommand{\cfc}[1]{\calligra{#1}\fontfamily{cmr}\selectfont{}} 

% Enumerate
\setlist[enumerate]{topsep=0pt,itemsep=-1ex,partopsep=1ex,parsep=1ex}

% Remove proofs by empty space
\NewEnviron{reviewmode}{%
    \let\visibleproof\proof
    \let\endvisibleproof\endproof
    \RenewEnviron{proof}{\phantom{}\\\ \\ \\}{}
    \BODY
    \let\proof\visibleproof
    \let\endproof\endvisibleproof
}

% Darkmode (black background, white text)
\newcommand{\darkmode}{\pagecolor{black}\color{white}}

% Enumerate with tab
\newenvironment{level}
{\addtolength{\itemindent}{2em}}
{\addtolength{\itemindent}{-2em}}

% Circle
\newcommand*\circled[1]{\tikz[baseline=(char.base)]{\node[shape=circle,draw,inner sep=0.5pt](char){#1};}}

% Roman numberals
\newcommand{\rom}{\romannumeral}

% Pgfplot setup
\pgfplotsset{compat=1.18}

% 3. Math

% Theorem styles
\theoremstyle{definition}
\newtheorem{definition}{Definition}[section]
\newtheorem{theorem}{Theorem}[section]
\newtheorem*{proposition}{Proposition}
\newtheorem*{lemma}{Lemma}
\newtheorem*{corollary}{Corollary}
\newtheorem*{example}{Example}
\newtheorem*{remark}{Remark}
\newtheorem*{notation}{Notation}
\newtheorem{questioninner}{Exercise}
\newenvironment{exercise}[1][]{%
    \ifx\relax#1\relax\else\renewcommand{\thequestioninner}{#1}\fi 
    \questioninner
}{%
}
\makeatletter % title: exercise, with customize []
\newenvironment{x}[1]{%
    \par\medskip\noindent\textbf{#1\@addpunct{.}}\hskip\labelsep
}{%
    \par 
}
\makeatother % general environment for any title you want

% Equation counter
\counterwithin*{equation}{section}
\counterwithin*{equation}{subsection}

% Quiver (Authors: varkor (https://github.com/varkor), AndréC (https://tex.stackexchange.com/users/138900/andr%C3%A9c))
\usetikzlibrary{calc}
\usetikzlibrary{decorations.pathmorphing}
\tikzset{curve/.style={settings={#1},to path={(\tikztostart)
    .. controls ($(\tikztostart)!\pv{pos}!(\tikztotarget)!\pv{height}!270:(\tikztotarget)$)
    and ($(\tikztostart)!1-\pv{pos}!(\tikztotarget)!\pv{height}!270:(\tikztotarget)$)
    .. (\tikztotarget)\tikztonodes}},
    settings/.code={\tikzset{quiver/.cd,#1}
        \def\pv##1{\pgfkeysvalueof{/tikz/quiver/##1}}},
    quiver/.cd,pos/.initial=0.35,height/.initial=0}
\tikzset{tail reversed/.code={\pgfsetarrowsstart{tikzcd to}}}
\tikzset{2tail/.code={\pgfsetarrowsstart{Implies[reversed]}}}
\tikzset{2tail reversed/.code={\pgfsetarrowsstart{Implies}}}
\tikzset{no body/.style={/tikz/dash pattern=on 0 off 1mm}}

% Change font (math) 
\newcommand{\bb}[1]{\mathbb{#1}}
\newcommand{\ca}[1]{\mathcal{#1}}
\newcommand{\fr}[1]{\mathfrak{#1}}

% Natural, rational, real, complex numbers, integers
\newcommand{\N}{\mathbb{N}}
\newcommand{\C}{\mathbb{C}}
\newcommand{\R}{\mathbb{R}}
\newcommand{\Q}{\mathbb{Q}}
\newcommand{\Z}{\mathbb{Z}}

% Topological space
\newcommand{\T}{\mathcal{T}}

% Fancy symmetric group and alternating group
\newcommand{\Sg}{\mathfrak{S}}
\newcommand{\Ag}{\mathfrak{A}}

% Symmetric group
\DeclareMathOperator{\Sym}{Sym}

% Uniqueness
\newcommand{\uni}{\exists\ \text{!}\ }

% Greek and Hebrew letters
\newcommand{\al}{\alpha}
\newcommand{\be}{\beta}
\newcommand{\ga}{\gamma}
\newcommand{\ep}{\epsilon}
\newcommand{\de}{\delta}
\newcommand{\si}{\sigma}
\newcommand{\la}{\lambda}
\newcommand{\ka}{\kappa}
\newcommand{\vt}{\vartheta}
\newcommand{\vp}{\varphi}
\newcommand{\ve}{\varepsilon}

% Arrows, maps, morphisms
\newcommand{\ua}{\uparrow}
\newcommand{\da}{\downarrow}
\newcommand{\Ra}{\Rightarrow}
\newcommand{\La}{\Leftarrow}
\newcommand{\Ua}{\Uparrow}
\newcommand{\Da}{\Downarrow}
\newcommand{\nRa}{\nRightarrow}
\newcommand{\nLa}{\nLeftarrow}
\newcommand{\hra}{\hookrightarrow}
\newcommand{\hla}{\hookleftarrow}
\newcommand{\lt}{\leadsto}
\newcommand{\mt}{\mapsto}
\newcommand{\rat}{\rightarrowtail}
\newcommand{\lat}{\leftarrowtail}
\DeclareMathOperator{\id}{id}
\newcommand{\bij}{\xrightarrow{\sim}} % right arrow with \sim on it

% Sets, inclusions
\newcommand{\sub}{\subset}
\newcommand{\sube}{\subseteq}
\newcommand{\supe}{\supseteq}
\newcommand{\nsub}{\centernot\subset}
\newcommand{\nsup}{\centernot\supset}
\newcommand{\nsube}{\centernot\subseteq}
\newcommand{\nsupe}{\centernot\supseteq}
\newcommand{\es}{\varnothing}
\newcommand{\sm}{\setminus}
\newcommand{\ps}{\mathscr{P}}

% Otimes, oplus
\newcommand{\ot}{\otimes}
\newcommand{\op}{\oplus}

% Sign of permutation
\DeclareMathOperator{\sgn}{sgn}

% Normal Subgroup
\newcommand{\nsg}{\trianglelefteq}

% Defined as
\newcommand{\defa}{\coloneqq}

% Semidirect product
\newcommand{\sdp}{\rtimes}

% Inverse
\newcommand{\inv}[1]{{#1}^{-1}}

% Union, intersection, disjoint union
\newcommand{\Cap}{\bigcap}
\newcommand{\Cup}{\bigcup}
\newcommand{\DU}{\bigsqcup}

% Modulo
\renewcommand{\mod}{\ \text{mod}\ }

% Conjugacy classes
\DeclareMathOperator{\Cl}{Cl}

% Holomorph
\DeclareMathOperator{\Hol}{Hol}

% Composition of functions
\newcommand{\comp}{~\raisebox{1pt}{\tikz \draw[line width=0.6pt] circle(1.1pt);}~}

% Galois group
\DeclareMathOperator{\gal}{Gal}

% Cardinality
\newcommand{\card}[1]{\lvert{#1}\rvert}

% Norm
\newcommand{\norm}[1]{\lVert{#1}\rVert}

% Partial order
\newcommand{\po}{\preceq}

% Groups generated by
\newcommand{\cyc}[1]{\langle{#1}\rangle}

% Spectrum of a ring
\DeclareMathOperator{\Spec}{Spec}

% Sylow groups
\DeclareMathOperator{\Syl}{Syl}

% Category
\newcommand{\iso}{\approx}
\DeclareMathOperator{\Aut}{Aut}
\DeclareMathOperator{\End}{End}
\DeclareMathOperator{\Hom}{Hom}
\DeclareMathOperator{\Inn}{Inn}
\DeclareMathOperator{\Out}{Out}
\DeclareMathOperator{\Iso}{Iso}
\DeclareMathOperator{\Ob}{Ob}
\newcommand{\cop}[1]{{#1}^{op}}

% Triangle 
\newcommand{\tri}{\triangle}

% Partial derivative
\newcommand{\pa}{\partial}

% 4. Physics & Chemistry

% Quantum: h-bar
\newcommand{\hb}{\hbar}

% Partial operator
\newcommand{\pr}{\partial}



\endinput % Download and input it using its path 
\end{verbatim}
\section{Title Page Setup}
After inserting the package, you should define the title and author name as follows:
\begin{verbatim}
    \begin{document}
        \def\htitle{Your Title} % replace ``Your Title'' with the title you want
        \def\hauthor{Your Name} % replace ``Your Name'' with the author name you want
        \hsetup % given the parameters, this should setup the title
    \end{document}
\end{verbatim}
\noindent
You can setup the table of contents by the code:
\begin{verbatim}
    \begin{document}
        \htoc
    \end{document}
\end{verbatim}
This will automatically generates a table of contents when you add a section to the document.
\section{Mainmatter of the Document}
Every page in the mainmatter has a header, which contains author name, title, and page number. Use the following code to setup:
\begin{verbatim}
    \begin{document}
        \hmain   
    \end{document}
\end{verbatim}
\section{An Example: This Demo}
This demo offers an easy example of how to use the style. Here is my code for this demo:
\begin{verbatim}
    \documentclass[10pt]{article} % The font size does not matter 
    %%% This is a scheme of a simple package. %%%
% version 1.0.6

% 1. Packages
\usepackage[T1]{fontenc}
\usepackage[hidelinks]{hyperref}
\usepackage[explicit]{titlesec}
\usepackage[utf8]{inputenc}
\usepackage{amsmath,amsthm,amssymb,amsfonts,mathrsfs,mathtools,nicematrix,chngcntr,centernot,ytableau,tikz-cd}
\usepackage{environ,setspace,geometry,enumerate,enumitem,blindtext,multicol,xcolor,calligra,graphicx,wrapfig,pgfplots,mdframed,tabularx,lipsum,comment,csquotes}
\usepackage{chemfig}

% 2. General Commands

% Enable useless warnings
% chktex-file 1 
% chktex-file 36 
% chktex-file 12 
% chktex-file 26 
% chktex-file 18

% Multicolumn setup
\setlength{\columnseprule}{1pt}
\def\columnseprulecolor{\color{black}}

% Geometry 
\geometry{letterpaper, margin=0.75in}
\setstretch{1.25}

% Newsection (shown in ToC, no counter)
\makeatletter
\newcommand\newsection[1]{%
  \section*{#1}%
  \addcontentsline{toc}{section}{#1}%
}
\makeatother

% Backslash
\newcommand{\bs}{\backslash}

% Hyperlink on ToC and section titles
\titleformat{\section}
{\normalfont\Large\bfseries}{\thesection}{1em}{\hyperlink{sec-\thesection}{#1}
\addtocontents{toc}{\protect\hypertarget{sec-\thesection}{}}}
\titleformat{name=\section,numberless}
{\normalfont\Large\bfseries}{}{0pt}{\hyperlink{toc}{#1}}

% Table of contents section only
\setcounter{tocdepth}{1}

% Changefont
\newcommand{\cfd}[1]{\fontfamily{pzc}\selectfont{#1}\fontfamily{cmr}\selectfont{}} 
\newcommand{\cfc}[1]{\calligra{#1}\fontfamily{cmr}\selectfont{}} 

% Enumerate
\setlist[enumerate]{topsep=0pt,itemsep=-1ex,partopsep=1ex,parsep=1ex}

% Remove proofs by empty space
\NewEnviron{reviewmode}{%
    \let\visibleproof\proof
    \let\endvisibleproof\endproof
    \RenewEnviron{proof}{\phantom{}\\\ \\ \\}{}
    \BODY
    \let\proof\visibleproof
    \let\endproof\endvisibleproof
}

% Darkmode (black background, white text)
\newcommand{\darkmode}{\pagecolor{black}\color{white}}

% Enumerate with tab
\newenvironment{level}
{\addtolength{\itemindent}{2em}}
{\addtolength{\itemindent}{-2em}}

% Circle
\newcommand*\circled[1]{\tikz[baseline=(char.base)]{\node[shape=circle,draw,inner sep=0.5pt](char){#1};}}

% Roman numberals
\newcommand{\rom}{\romannumeral}

% Pgfplot setup
\pgfplotsset{compat=1.18}

% 3. Math

% Theorem styles
\theoremstyle{definition}
\newtheorem{definition}{Definition}[section]
\newtheorem{theorem}{Theorem}[section]
\newtheorem*{proposition}{Proposition}
\newtheorem*{lemma}{Lemma}
\newtheorem*{corollary}{Corollary}
\newtheorem*{example}{Example}
\newtheorem*{remark}{Remark}
\newtheorem*{notation}{Notation}
\newtheorem{questioninner}{Exercise}
\newenvironment{exercise}[1][]{%
    \ifx\relax#1\relax\else\renewcommand{\thequestioninner}{#1}\fi 
    \questioninner
}{%
}
\makeatletter % title: exercise, with customize []
\newenvironment{x}[1]{%
    \par\medskip\noindent\textbf{#1\@addpunct{.}}\hskip\labelsep
}{%
    \par 
}
\makeatother % general environment for any title you want

% Equation counter
\counterwithin*{equation}{section}
\counterwithin*{equation}{subsection}

% Quiver (Authors: varkor (https://github.com/varkor), AndréC (https://tex.stackexchange.com/users/138900/andr%C3%A9c))
\usetikzlibrary{calc}
\usetikzlibrary{decorations.pathmorphing}
\tikzset{curve/.style={settings={#1},to path={(\tikztostart)
    .. controls ($(\tikztostart)!\pv{pos}!(\tikztotarget)!\pv{height}!270:(\tikztotarget)$)
    and ($(\tikztostart)!1-\pv{pos}!(\tikztotarget)!\pv{height}!270:(\tikztotarget)$)
    .. (\tikztotarget)\tikztonodes}},
    settings/.code={\tikzset{quiver/.cd,#1}
        \def\pv##1{\pgfkeysvalueof{/tikz/quiver/##1}}},
    quiver/.cd,pos/.initial=0.35,height/.initial=0}
\tikzset{tail reversed/.code={\pgfsetarrowsstart{tikzcd to}}}
\tikzset{2tail/.code={\pgfsetarrowsstart{Implies[reversed]}}}
\tikzset{2tail reversed/.code={\pgfsetarrowsstart{Implies}}}
\tikzset{no body/.style={/tikz/dash pattern=on 0 off 1mm}}

% Change font (math) 
\newcommand{\bb}[1]{\mathbb{#1}}
\newcommand{\ca}[1]{\mathcal{#1}}
\newcommand{\fr}[1]{\mathfrak{#1}}

% Natural, rational, real, complex numbers, integers
\newcommand{\N}{\mathbb{N}}
\newcommand{\C}{\mathbb{C}}
\newcommand{\R}{\mathbb{R}}
\newcommand{\Q}{\mathbb{Q}}
\newcommand{\Z}{\mathbb{Z}}

% Topological space
\newcommand{\T}{\mathcal{T}}

% Fancy symmetric group and alternating group
\newcommand{\Sg}{\mathfrak{S}}
\newcommand{\Ag}{\mathfrak{A}}

% Symmetric group
\DeclareMathOperator{\Sym}{Sym}

% Uniqueness
\newcommand{\uni}{\exists\ \text{!}\ }

% Greek and Hebrew letters
\newcommand{\al}{\alpha}
\newcommand{\be}{\beta}
\newcommand{\ga}{\gamma}
\newcommand{\ep}{\epsilon}
\newcommand{\de}{\delta}
\newcommand{\si}{\sigma}
\newcommand{\la}{\lambda}
\newcommand{\ka}{\kappa}
\newcommand{\vt}{\vartheta}
\newcommand{\vp}{\varphi}
\newcommand{\ve}{\varepsilon}

% Arrows, maps, morphisms
\newcommand{\ua}{\uparrow}
\newcommand{\da}{\downarrow}
\newcommand{\Ra}{\Rightarrow}
\newcommand{\La}{\Leftarrow}
\newcommand{\Ua}{\Uparrow}
\newcommand{\Da}{\Downarrow}
\newcommand{\nRa}{\nRightarrow}
\newcommand{\nLa}{\nLeftarrow}
\newcommand{\hra}{\hookrightarrow}
\newcommand{\hla}{\hookleftarrow}
\newcommand{\lt}{\leadsto}
\newcommand{\mt}{\mapsto}
\newcommand{\rat}{\rightarrowtail}
\newcommand{\lat}{\leftarrowtail}
\DeclareMathOperator{\id}{id}
\newcommand{\bij}{\xrightarrow{\sim}} % right arrow with \sim on it

% Sets, inclusions
\newcommand{\sub}{\subset}
\newcommand{\sube}{\subseteq}
\newcommand{\supe}{\supseteq}
\newcommand{\nsub}{\centernot\subset}
\newcommand{\nsup}{\centernot\supset}
\newcommand{\nsube}{\centernot\subseteq}
\newcommand{\nsupe}{\centernot\supseteq}
\newcommand{\es}{\varnothing}
\newcommand{\sm}{\setminus}
\newcommand{\ps}{\mathscr{P}}

% Otimes, oplus
\newcommand{\ot}{\otimes}
\newcommand{\op}{\oplus}

% Sign of permutation
\DeclareMathOperator{\sgn}{sgn}

% Normal Subgroup
\newcommand{\nsg}{\trianglelefteq}

% Defined as
\newcommand{\defa}{\coloneqq}

% Semidirect product
\newcommand{\sdp}{\rtimes}

% Inverse
\newcommand{\inv}[1]{{#1}^{-1}}

% Union, intersection, disjoint union
\newcommand{\Cap}{\bigcap}
\newcommand{\Cup}{\bigcup}
\newcommand{\DU}{\bigsqcup}

% Modulo
\renewcommand{\mod}{\ \text{mod}\ }

% Conjugacy classes
\DeclareMathOperator{\Cl}{Cl}

% Holomorph
\DeclareMathOperator{\Hol}{Hol}

% Composition of functions
\newcommand{\comp}{~\raisebox{1pt}{\tikz \draw[line width=0.6pt] circle(1.1pt);}~}

% Galois group
\DeclareMathOperator{\gal}{Gal}

% Cardinality
\newcommand{\card}[1]{\lvert{#1}\rvert}

% Norm
\newcommand{\norm}[1]{\lVert{#1}\rVert}

% Partial order
\newcommand{\po}{\preceq}

% Groups generated by
\newcommand{\cyc}[1]{\langle{#1}\rangle}

% Spectrum of a ring
\DeclareMathOperator{\Spec}{Spec}

% Sylow groups
\DeclareMathOperator{\Syl}{Syl}

% Category
\newcommand{\iso}{\approx}
\DeclareMathOperator{\Aut}{Aut}
\DeclareMathOperator{\End}{End}
\DeclareMathOperator{\Hom}{Hom}
\DeclareMathOperator{\Inn}{Inn}
\DeclareMathOperator{\Out}{Out}
\DeclareMathOperator{\Iso}{Iso}
\DeclareMathOperator{\Ob}{Ob}
\newcommand{\cop}[1]{{#1}^{op}}

% Triangle 
\newcommand{\tri}{\triangle}

% Partial derivative
\newcommand{\pa}{\partial}

% 4. Physics & Chemistry

% Quantum: h-bar
\newcommand{\hb}{\hbar}

% Partial operator
\newcommand{\pr}{\partial}



\endinput
    \begin{document}
        \def\htitle{Demo of Hassium Style}
        \def\hauthor{Hassium}
        \hsetup\
        \htoc\
        \hmain\
    \end{document}
\end{verbatim}
\section{Setup in Geometry}
There are some commands that adjust the geometry of the document:
\begin{verbatim}
    \geometry{letterpaper, margin=0.75in}
    \setstretch{1.25} % spacing
    \setlength{\headheight}{13pt}
\end{verbatim}
\section{More on Table of Contents}
You can add descriptions to each section and the description will appear in the table of contents, directly below the section name: 
\begin{verbatim}
    \section{This is a Sample Section} 
    \descr{This is a description to the section} 
\end{verbatim}
\noindent
The table of contents only shows the section names, but no subsections and numberless sections. If you want a numberless section in the table of contents, use the ``newsection'' command:
\begin{verbatim}
    \newsection{This is a numberless section} 
\end{verbatim}
\noindent
Note that the section names in the table of contents are hyperlinks; click on any section name to navigate directly to that section. You can do the converse to navigate to the first page as well.
\section{Darkmode}
Darkmode command changes the background color to black and the text to white. The normal mode is used to end the darkmode. Use the commands by:
\begin{verbatim}
    \begin{document}
        \darkmode 
        \normalmode
    \end{document}
\end{verbatim}
\section{Other Environments and Commands}
The line-spacing in ``enumerate'' environment is changed:
\begin{verbatim}
    \setlist[enumerate]{topsep=0pt,itemsep=-1ex,partopsep=1ex,parsep=1ex}
\end{verbatim}
\noindent
The ``level'' environment is used in ``enumerate'' environment, consider the following code:
\begin{verbatim}
    \begin{enumerate}
        \item This is the first line.
        \begin{level}
            \item This is the second line.
            \begin{level}
                \item This is the third line.
            \end{level}
            \item This is another line.
        \end{level}
    \end{enumerate}
\end{verbatim}
\noindent
This code gives:
\begin{enumerate}
    \item This is the first line.
    \begin{level}
        \item This is the second line.
        \begin{level}
            \item This is the third line.
        \end{level}
        \item This is another line.
    \end{level}
\end{enumerate}
\noindent
The command ``circled'' draws a small circle and you can add something inside the circle:
\begin{verbatim}
    \circled{1}
\end{verbatim}
\noindent
The output is $\circled{1}$. You can write any romam numerals by:
\begin{verbatim}
    \rom2024 % replace 2024 by any number you want
\end{verbatim}
\noindent
There are two simple commands for hand-written fonts:
\begin{verbatim}
    \cfd{font 1}
    \cfc{font 2}
\end{verbatim}
The outputs are \cfd{font 1} and \cfc{font 2}.
\section{Quiver}
Quiver is done by \href{https://github.com/varkor}{varkor} and \href{https://tex.stackexchange.com/users/138900/andr%C3%A9c}{AndréC}, check their github for more information. I include quiver to draw curve arrows in a commutative diagram. To draw a diagram with quiver, check this \href{https://q.uiver.app/}{website}. An example is given below:
\begin{verbatim}
    % chktex-file 15 % the three lines enables useless warnings
    % chktex-file 17
    % chktex-file 18
    \begin{center}
        \begin{tikzcd}
            Hello &&&& World \\
            \\
            \\
            && Hassium
            \arrow["\shortmid"{marking}, curve={height=-6pt}, tail reversed, from=1-1, to=1-5]
            \arrow[curve={height=6pt}, squiggly, from=1-1, to=4-3]
            \arrow[curve={height=-6pt}, dashed, hook', from=1-5, to=4-3]
        \end{tikzcd}
    \end{center}
\end{verbatim}
The diagram looks like:
\begin{center}
    $\begin{tikzcd}
        Hello &&&& World \\
        \\
        \\
        && Hassium
        \arrow["\shortmid"{marking}, curve={height=-6pt}, tail reversed, from=1-1, to=1-5]
        \arrow[curve={height=6pt}, squiggly, from=1-1, to=4-3]
        \arrow[curve={height=-6pt}, dashed, hook', from=1-5, to=4-3]
    \end{tikzcd}$
\end{center}
\section{Theorem Styles}
\section{Invisible Proofs}
The environment ``reviewmode'' is originally done by my friend \href{https://github.com/ETwilight}{ETwilight}. It replaces your ``proof'' environment by three empty lines:
\begin{verbatim}
    \begin{reviewmode}
        \begin{proof}
            The proof is trivial.
        \end{proof}
    \end{reviewmode}
\end{verbatim}
\section{Simple Commands in Math Mode}
I will give a table of all commands in math mode.
\end{document}